
\chapter{Arquitetura aplicada: 4chan.org}

Com o objetivo de instanciar um exemplo concreto da estratégia de transição que propomos, implementaremos as ferramentas acima propostas de forma que permitam o armazenamento e compartilhamento de conteúdo do 4chan de forma descentralizada.
Brevemente, essas ferramentas serão:
\begin{enumerate*}[label=(\arabic*)]
    \item Um \textit{scraper} confiável e de de alto desempenho chamado \texttt{archive-chan},
    \item um módulo de \textit{networking} capaz de localizar quais peers possuem quais dados, e
    \item um módulo de criptografia para assinar e verificar assinaturas, a fim de montarmos uma \textit{internet of trust}. 
\end{enumerate*}

\section{Requisitos da Solução}

Neste capítulo serão apresentados os Requisitos Funcionais e Requisitos Não Funcionais da solução para separar a camada de dados da camada de apresentação do 4chan.

Segundo Sommerville (2011, p. 57)~\cite{SOMMERVILLE1}, os requisitos de um sistema são:

\begin{directcite}
as descrições do que o sistema deve fazer, os serviços que
oferece e as restrições a seu funcionamento.
Esses requisitos refletem as necessidades dos clientes para
um sistema que serve a uma finalidade determinada, como controlar
um dispositivo, colocar um pedido ou encontrar informações.
\end{directcite}

\subsection{Requisitos Funcionais}

Os requisitos funcionais de um sistema representam as necessidades que o sistema deverá suprir e todas as funções que o sistema deve ter.
Sobre os requisitos funcionais de uma solução, Sommerville (2011, p. 59) considera que~\cite{SOMMERVILLE1}:

\begin{directcite}
São declarações de serviços que o sistema deve fornecer,
de como o sistema deve reagir a entradas específicas e
de como o sistema deve se comportar em determinadas situações.
Em alguns casos, os requisitos funcionais também
podem explicitar o que o sistema não deve fazer.
\end{directcite}

De acordo com o nosso entendimento, a solução possui os seguintes requisitos funcionais.

\begin{enumerate}
    \item Módulo de \textit{scraping}:
    \begin{enumerate}[label*=\arabic*.]
        \item baixar todo o texto de uma thread.
            \subitem escolher uma thread passando sua URL por argumento.
            \subitem saber quando o download foi concluído.
        \item baixar toda a mídia (fotos, vídeos) de uma thread.
            \subitem escolher uma thread passando sua URL por argumento.
            \subitem monitorar o progresso do download.
            \subitem poder interromper um download.
            \subitem continuar um download de onde parou.
        \item baixar todas threads de um board.
            \subitem escolher um board passando seu nome como argumento.
            \subitem monitorar quais threads estão sendo baixadas.
            \subitem baixar threads utilizando computação paralela.
            \subitem interromper download.
            \subitem retomar download.
        \item reconhecer quando uma thread estiver fechada e marcá-la como tal para evitar desperdício de recursos.
            \subitem reconhecer threads arquivadas.
            \subitem reconhecer threads apagadas.
            \subitem marcar como tal no banco de dados.
        \item monitorar threads por novos posts e baixá-los assim que postados.
            \subitem opcionalmente receber um argumento que defina de quanto em quanto tempo checar por novos posts numa thread.
        \item monitorar boards por novos posts e adicioná-las a fila de monitoramento de threads assim que criadas.
            \subitem opcionalmente receber um argumento que defina de quanto em quanto tempo checar por novas threads num board.
        \item converter os dados baixados em um formato padronizado.
            \subitem opcionalmente receber um argumento para exportar threads.
            \subitem permitir que o usuário especifique qual dos padrões implementados ele deseja.
            \subitem utilizar computação paralela na exportação.
    \end{enumerate}
    \item Módulo de \textit{networking}:
    \begin{enumerate}[label*=\arabic*.]
        \item Requisitar conteúdo para outros peers.
        \item Servir conteúdo local para outros peers.
    \end{enumerate}
    \item Módulo de criptografia:
    \begin{enumerate}[label*=\arabic*.]
        \item Criar um novo par de chaves (privada e pública) criptográficas.
        \item Escolher em quais peers confio para baixar conteúdo que não esteja disponível no servidor do 4chan.
        \item Exportar minha chave pública em cleartext para compartilhamento.
        \item Assinar arquivos com minha chave privada.
        \item Verificar se assinatura de arquivos correspondem a alguma chave pública que tenho marcada como confiável.
    \end{enumerate}
\end{enumerate}

\subsection{Requisitos Não Funcionais}

Sobre os requisitos não funcionais de uma solução, Sommerville (2011, p. 59) considera que~\cite{SOMMERVILLE1}:

\begin{directcite}
Os requisitos não funcionais, como o nome sugere, são requisitos
que não estão diretamente relacionados com os serviços específicos
oferecidos pelo sistema a seus usuários.
Eles podem estar
relacionados às propriedades emergentes do sistema, como
confiabilidade, tempo de resposta e ocupação de área.
\end{directcite}

Em outras palavras, os requisitos não funcionais de um sistema representam as necessidades internas do sistema, necessidades estas que devem ser de compreensão do desenvolvedor, resumindo-se aos itens de segurança, usabilidade, confiabilidade, desempenho, hardware e software.
De acordo com a análise realizada, as ferramentas possuem os seguintes requisitos não funcionais:

\begin{enumerate}
    \item \textbf{Desempenho}: Ter desempenho em termos de \textit{response time} similares às redes da web 2.0 (média inferior a 300ms, quanto menor melhor).
    \item \textbf{Tecnologia}: Empregar um formato de dados baseado em padrões livres (\textit{free software}) e código aberto (\textit{open-source}) para facilitar extensão e tornar resistente ao futuro\footnotemark.
    \footnotetext{
        \textit{Future-proofing} em desenvolvimento de software tem um escopo bem abrangente.
        Mas esse ponto é focado em permitir que minimizar as dificuldades que alguém encontraria para interagir com dados do nosso programa caso encontrasse uma pasta criada por ele daqui 25 anos, por exemplo.
    }
    \item \textbf{Desempenho}: Utilizar estruturas de dados que permitam downloads de granularidade pequena (não ser necessário baixaar um \textit{board} inteiro para baixar uma \textit{thread}).
    \item \textbf{Segurança}: Utilizar estruturas de dados que possam ser replicadas e atualizadas independentemente e paralelamente numa rede peer-to-peer, sem a necessidade de coordenação entre as réplicas, nem uma entidade central (CRDTs, conflict-free replicated data types\footnote{\url{https://en.wikipedia.org/wiki/Conflict-free_replicated_data_type}}).
    \item \textbf{Desempenho}: O formato de dados deve evitar duplicação de conteúdo, a fim de economizar recursos de armazenamento.
    \item \textbf{Ambiente}: Ser compatíveis com distribuições GNU/Linux.
    \item \textbf{Usabilidade}: Seguir os princípios da filosofia UNIX\footnote{\url{https://en.wikipedia.org/wiki/Unix_philosophy}}, especialmente no que diz respeito a modularidade e responsabilidade única.
\end{enumerate}

% \begin{figure}[H]
%     \centering
%     \includesvg[width=1.1\textwidth]{fig/RequisitosNaoFuncionais.svg}
%     \caption[Requisitos Não-Funcionais]{\label{fig:Requisitos_Nao_Funcionais}
%         Requisitos Não-Funcionais\\
%         Fonte: os autores.
%     }
% \end{figure}

% {
% \renewcommand{\arraystretch}{3}
% \begin{table}[!htbp]
%     \centering
%     \begin{tabularx}{\textwidth}{|X|X|X|}
%         \hline
%         \centering \textbf{ID Requisito}  & \centering  \textbf{Tipo Requisito} & \textbf{Descrição} \\
%         \hline
%         \textbf{RNF01}       & Desempenho       & \textit{Response times} com média inferior a 300ms.               \\
%         \hline
%         RNF02   & Tecnologia      &        Empregar um formato de dados baseado em padrões livres e open-source.     \\
%         \hline
%         RNF03 &   Desempenho    & Utilizar estruturas de dados que permitam downloads de granularidade pequena.         \\
%         \hline
%         RNF04     &   Segurança         & Utilizar CRDTs.    \\
%         \hline
%         RNF05 &    Desempenho     & O formato de dados deve evitar duplicação de conteúdo, a fim de economizar recursos de armazenamento.   \\
%         \hline
%         RNF06     &  Ambiente         & Ser compatíveis com distribuições GNU/Linux.               \\
%         \hline
%         RNF07  &    Usabilidade        & Respeitar a filosofia UNIX.   \\
%         \hline
%     \end{tabularx}
%     \caption{\label{tab:non-func-reqs}
%         Cronograma de atividades.\\
%         Fonte: os autores.
%     }
% \end{table}}

\section{Tecnologias que utilizaremos}

Nesta seção, falaremos sobre as tecnologias que utlizamos para desenvolver as ferramentas.

\subsection{Python multiprocessing}

Um web-scraper precisa se comunicar com servidores web pela internet para requisitar conteúdo e precisa aguardar o conteúdo ser baixado para interagir com ele.
Ambas operações de IO são bloqueantes, então para melhor aproveitar os recursos computacionais disponíveis, utilizaremos uma estratégia de computação paralela estabelecida: Uma fila de tarefas e $N$ processos consumindo tarefas dessa fila.
Em python, há uma biblioteca de paralelismo multiprocesso chamada \texttt{multiprocessing}\footnote{\url{https://docs.python.org/3/library/multiprocessing.html}} que aproveitaremos para abstrair a parte de sistemas operacionais do código do nosso scraper.

\subsection{Python requests}

A biblioteca padrão de python para lidar com URLs (realizar e gerenciar requisições HTTP é apenas um subset de suas features), \texttt{urllib}, é de relativamente baixo nível e, na opinião dos autores, mal arquitetada.
Optamos, portanto, por implementarmos nosso cliente HTTP com a biblioteca third-party chamada \texttt{requests}\footnote{\url{https://docs.python-requests.org/en/master/}} que é focada apenas em HTTP+JSON, o que simplifica consideravelmente o seu uso, por não ter as abstrações agnósticas a protocolo e a corpo de requisições e respostas.

\subsection{Python setuptools}

Como parte da estratégia de distribuição, decidimos utilizar o PIP (gerenciador de pacotes padrão de python).
Existem várias bibliotecas de empacotamento para distribuição com PIP (como \texttt{poetry}, \texttt{flit}, e \texttt{distutils}), mas escolhemos \texttt{setuptools}\footnote{\url{https://setuptools.readthedocs.io/en/latest/}} por ser a mais difundida e melhor documentada dentre elas.

\subsection{Python toolz}

Devido ao viés negativo que o criador, mantenedor, e ditador benevolente vitalício de python, Guido van Rossum, tem sobre programação funcional, o pacote \texttt{functools} incluso na biblioteca padrão deixa muito a desejar em termos de manipulação de dados com funções de alta ordem.
Então, como já é costume em python, optamos por utilizar uma biblioteca third-party.
Escolhemos uma chamada \texttt{toolz}\footnote{\url{https://github.com/pytoolz/toolz}} que oferece maior flexibilidade, sintaxes mais sucintas, e melhor performance que sua contraparte da biblioteca padrão.

\subsection{\label{subsec:ipfs}InterPlanetary File System}

O IPFS é um sistema de arquivos distribuído e protocolo de hipermedia~\cite{benet2014ipfs}.
Ele provê um modelo de armazenamento de objetos em blocos, com hyperlinks endereçados por conteúdo; tudo através do seu formato de dados central chamado IPLD (InterPlanetary Linked Data) que será explicado em maior detalhe na subseção~\ref{subsubsec:ipld}.
Juan Benet, autor do IPFS, frequentemente o descreve~\cite{github:ipfs} como um único swarm BitTorrent\ref{bittorent-swarm} com peers trocando objetos git entre si usando self-certifying paths para se referir a arquivos e usando um DHT Kademlia; representamos essas analogias no diagrama da figura~\ref{fig:ipfs-stack-equivalences}.
Isso é simples, informal, e incompleto, mas é uma analogia poderosa que captura as mecânicas essenciais do IPFS.
\longfootnote[bittorent-swarm]{
    Em BitTorrent, um swarm é o conjunto de peers (incluindo seeders) compartilhando algum torrent específico.
}

\begin{figure}[htb]
    \centering
    \includegraphics[width=\textwidth]{fig/ipfs-stack-equivalences.pdf}
    \caption[IPFS stack equivalences]{
        Modelo de rede do IPFS com as tecnologias em que ele se inspira em cada camada. Em rosa as camadas abstraídas pela IPLD e em azul pela libp2p.\\
        Fonte: os autores.
    }
    \label{fig:ipfs-stack-equivalences}
\end{figure}

Outro ponto que Juan Benet frequentemente ressalta é o IPFS como um protocolo de thin-waist (conceito expandido na subseção~\ref{subsec:thin-waist} (cintura-fina, ampulheta);
querendo dizer que a camada da aplicação e a camada de networking podem modular arbitrariamente, contanto que o que há no meio (as estruturas de dados da IPLD) seja padronizado todo o resto se comunicará e funcionará.
O que pode ser visto na figura~\ref{fig:ipfs-stack-thin-waist}.

\begin{figure}[htb]
    \centering
    \includegraphics[width=\textwidth]{fig/ipfs-stack-thin-waist.pdf}
    \caption[IPFS's hourglass network stack]{
        Modelo de rede do IPFS com os protocolos suportados demonstrando a cintura fina que são os formatos de dados definidos pela IPLD. Defaults destacados em vermelho.\\
        Fonte: os autores.
    }
    \label{fig:ipfs-stack-thin-waist}
\end{figure}

O IPFS não tem nenhum ponto central de falha; nodos não precisam confiar uns nos outros e todos os peers são independentes.
A rede é extremamente tolerante a partições e oferece alta disponibilidade, ao custo de garantir consistência, entretanto é arquitetada para consistência eventual (sempre recuperar com sucesso após partições).
Nos próximos parágrafos, detalharemos os subsistemas relevantes do IPFS.

\subsubsection{\label{subsubsec:ipld}IPLD}

O IPLD é um formato para estruturas de dados distribuídas que sejam universalmente endereçáveis e \textit{linkable}~\cite{github:ipld} baseado em apontamentos por hashes.
Todos dados do IPFS são armazenados e transferidos nesse formato.
Resumidamente, é um Merkle DAG onde os links entre os objetos (nodos do grafo) são hashes criptográficas do conteúdo dos objetos para que ele apontam.
Uma implementação de um formato de dados usando IPLD permite que usuários naveguem entre os dados usando \textit{paths} estilo UNIX; o que é um modelo de dados universal, prático, flexível, e planejado para ser adequadamente representável em JSON (embora tenha serializadores para vários formatos populares, e.g.: CBOR, Protobug, e RDF).
Enfatizamos que essa navegação se dá \textit{seamlessly} mesmo através de merkle trees de aplicações distintas (por isso a representação da figura~\ref{fig:ipld-links}, por exemplo, poderíamos partir de um repositório Git e abrir um link para uma transação Bitcoin que contém um link para uma foto no IPFS.
Essas estruturas permitem que desenvolvedores façam com dados e mídia o que URLs fizeram com HTML e páginas da Web.
Também é importante enfatizar que nada é perdido, em termos de funcionalidade, quando usamos esse formato de dados.

\begin{figure}[H]
    \centering
    \includegraphics[width=0.5\textwidth]{fig/ipld-links.pdf}
    \caption[IPFS's IPLD merkle forest]{
        Material oficial da Protocol Labs exemplificando uma coleção de merkle trees/dags (Git, Ethereum, Bitcoin, e IPFS) conectados através da IPLD.\\
        Fonte: \url{/ipfs/QmXFhzGYF27zvjNxbJNcfn226ZkJpRg2sQGRgK7JKdCKje/img/ipld.svg}.
    }
    \label{fig:ipld-links}
\end{figure}

\subsubsection{\label{subsubsec:libp2p}libp2p}

Uma coleção crescente, modular, flexível, e extensível de protocolos de rede peer-to-peer (ver figura~\ref{fig:libp2p-modularity}) cujo objetivo é interoperabilidade multiplataforma preparada para lidar com cenários de alta-latência e upgrades futuras.
Protocolos para encontrar peers, conectar com eles, localizar conteúdo, e transferí-lo.
Tudo enquanto não dependendo de registros centralizados e terceiros em nenhum passo, garantindo a segurança e descentralização da rede, permitindo que usuários trabalhem offline e empresas nas suas LANs.
Outro esforço relacionado à segurança é sempre realizar conexões criptografadas por padrão.

\begin{figure}[htbp]
    \centering
    \includegraphics[width=0.9\textwidth]{fig/libp2p-modularity.png}
    \caption[IPFS's libp2p: a modular network stack]{
        Material oficial da Protocol Labs demonstrando a modularidade da libp2p.
        Legenda: roxo, Content Routing; azul, Peer Routing; amarelo, Discovery; verde, Transports; vermelho, NAT Traversal.\\
        Fonte: \url{/ipfs/QmdzerM4fsnNGVf5jnogxu6QpXQa5rHDK87oHStC5xGCS4/img/boxdiagram.png}
    }
    \label{fig:libp2p-modularity}
\end{figure}

Um dos principais objetivos de desenvolver a \texttt{libp2p} separadamente do IPFS é para garantir que futuros desenvolvedores de sistemas distribuídos não precisem ``reinventar a roda'' no futuro próximo.
Até então, todos grandes projetos da web 3.0 (como BitTorrent, DAT, e todas cryptocurrencies) tiveram que escrever seu código de networking praticamente do zero.

\subsection{GNU Privacy Guard}

GNU Privacy Guard, também conhecido como GnuPG e GPG\footnote{\url{gnupg.org/}}, é uma implementação livre do padrão OpenPGP definido pela RFC4880.
GnuPG permite criptografar e assinar dados, gerenciar chaves de diferentes algoritmos (incluindo o estado da arte de criptografia assimétrica), assim como enviar e adquirir chaves públicas de diversos tipos de diretórios e \textit{key servers}.
GnuPG tem sua principal interface, seguindo tradição UNIX, na linha de comando permitindo fácil integração com outras aplicações.
Nós o escolhemos também por sua ubiquidade em sistemas compatíveis com POSIX.

\section{Implementação}

Os esforços serão concentrados na implementação do scraper, pois é o mais relevante para fins de demonstração do conceito e de contribuição open-source.
O IPFS, desenvolvido pela Protocol Labs, aliviará muitas das nossas preocupações de implementação no tocante a questão de \textit{networking}; permitindo-nos avançar bastante realizando apenas de integrações simples com a API da libp2p e da IPLD e escrevendo \textit{wrappers} de alto nível para o GnuPG no módulo de criptografia e o componente de grafo da nossa internet de confiança, como pode ser visto no diagrama de classes da figura~\ref{fig:internet-of-trust-class-diagram}.

\begin{figure}[htb]
    \centering
    \includegraphics[width=\textwidth]{fig/internet-of-trust-class-diagram-highlighted.pdf}
    \caption[Ferramenta 3: internet of trust - diagrama de classes]{
        Diagrama de classes representando a arquitetura da ferramenta proposta na seção~\ref{sec:int-of-trust}. Tipos representados em python.\\
        Fonte: os autores.
    }
    \label{fig:internet-of-trust-class-diagram}
\end{figure}


\subsection{Modelo de dados}

Na figura~\ref{fig:er-4chan}, dispomos um diagrama entidade-relacional com um esboço do modelo de dados do 4chan.
Simplificamos os atributos e não incluímos entidades que não fossem relevantes para o entendimento do modelo de dados.
De forma resumida, cada board possui $N$ threads e cada thread possui $N$ posts.

\begin{figure}[H]
    \centering
    \begin{tikzpicture}[auto,node distance=1.5cm]
        \node[entity] (board_node) {Boards}
            [grow=up,sibling distance=3cm]
            child {node[attribute] {Name}}
            child {node[attribute] {Abbreviation}};
        \node[relationship] (board_thread) [below right = of board_node] {have};
        \node[entity] (thread_node) [above right = of board_thread]	{Threads}
            child {node[attribute] {id}};
        \path (board_thread) edge node {1} (board_node) edge node {N} (thread_node);
        \node[entity] (post_node) [above = of thread_node] {Posts}
            [grow=up,sibling distance=3cm]
            child[grow=left,level distance=3cm] {node[attribute] {id}}
            child {node[attribute] {Author name}}
            child {node[attribute] {Media}}
            child {node[attribute] {Text}};
        \node[relationship] (thread_post) [above right = of thread_node] {have};
        \path (thread_post) edge node {1} (thread_node) edge node {N} (post_node);
    \end{tikzpicture}
    \caption[Diagrama ER do modelo de dados do 4chan]{
        Diagrama entidade-relacional demonstrando um modelo de dados de alto nível do 4chan.\\
        Fonte: os autores.
    }
    \label{fig:er-4chan}
\end{figure}

\subsection{Banco de dados}

Para o nosso banco de dados, escolhemos o que há de mais portátil e acessível: o sistema de arquivos.
Os dados colaboram para isso, pois não têm nenhuma relação com cardinalidade que nos ``impeça'' de modelá-los denormalizados num banco de dados \textit{NoSQL}, então assim o fizemos.
Como pode ser visto na árvore de diretórios na figura~\ref{fig:archive-chan-file-system-db}, temos:

\begin{itemize}
    \item Uma pasta raiz com $N$ subpastas, cada um correspondendo a um board.
    \item Cada pasta de board, identificada pela abreviação única do board, possui $N$ subpastas e cada uma dessas corresponde a uma thread.
    \item Cada pasta de thread possui 
        \subitem uma subpasta \code{media} onde todas imagens e vídeos da thread são armazenados;
        \subitem um arquivo \code{thread.json} onde todos dados pertinentes àquela thread são armazenados.
\end{itemize}

\begin{figure}[htbp]
    \centering
    \includegraphics[width=0.5\textwidth]{fig/archive-chan-file-system-db.pdf}
    \caption[Scraper File System Dabatase model]{
        Modelo do banco de dados NoSQL (árvore do sistema de arquivos) criado pelo web-scraper que estamos estamos propondo.\\
        Fonte: os autores.
    }
    \label{fig:archive-chan-file-system-db}
\end{figure}
