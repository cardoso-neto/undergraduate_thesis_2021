\chapter{Modelagem do Sistema}

Este capítulo apresenta a modelagem completa das ferramentas
propostas, buscando o esclarecimento de suas funcionalidades e características
através de sua abstração.

\section{Diagramas de Atividades}

Nesta seção serão apresentados os diagramas de atividade das ferramentas quem tem contato direto com o usuário(\textit{Scraper} e Rede \textit{Peer-to-Peer}), com foco nas atividades que serão executadas no funcionamento destas ferramentas.

\subsection{Extração de Dados\label{subsec:extracao_de_dados}}

Esse processo realiza a extração de dados da web 2.0, esse é o método principal para a migração de dados da web 2.0 para a web 3.0, as principais atividade do processo podem ser vistas no diagrama de atividades representado na figura ~\ref{fig:extracao_de_dados}.

\begin{figure}[htb]
    \centering
    \includegraphics[width=\textwidth]{fig/diagrama-extracao-dados.pdf}
    \caption[Activity diagram: data extraction]{
        Diagrama de Atividades: Extração de Dados.\\
        Fonte: os autores.
    }
    \label{fig:extracao_de_dados}
\end{figure}

Esse processo já está descrito a partir da rede social da web 2.0 que será utilizada para a aplicação das ferramentas deste trabalho, assim além
do usuário e da ferramenta \textit{scraper} temos a API do \textit{4chan}. Por estarmos utilizando o \textit{4chan} que é uma plataforma que não necessita o cadastro de um usuário não existe o passo para a autenticação junto a API, para outras plataformas como \textit{Facebook} e \textit{Twitter} necessitaríamos um passo a mais e a utilização de um \textit{token} nas chamadas a API.

O processo começa com a etapa onde o usuário indica qual o dado deseja salvar em seu computador; Exemplo: \textit{board}, \textit{thread}, comentário (Todos exemplos de postagens do \textit{4chan}). E depois o
formato que deseja que eles sejam salvos; Exemplo: XML, JSON, TXT. Após essa primeira etapa o \textit{scraper} recebe esses dados e cria uma requisição para a API do \textit{4chan} e fica no aguardo da
reposta, recebido a resposta o \textit{scraper} formata os dados conforme os parametros que o usuário especificou e salva os dados no local indicado pelo usuário.

\subsection{Solicitação de Dados pela Rede P2P\label{subsec:Solicitacao_Dados}}
Este processo realiza a comunicação e a troca de dados dentro da rede P2P, uma fez extraído os dados da web 2.0 esse vai ser o novo formato de obtenção e troca de dados sem a necessidade do 
do \textit{4chan}. A partir deste processo o usuário fica independente das plataformas da web 2.0. as principais atividade do processo podem ser vistas no diagrama de atividades representado na figura~\ref{fig:Solicitacao_Dados}.

\begin{figure}[htb]
    \includegraphics[width=\textwidth]{fig/activity-diagram-p2p-network.pdf}
    \caption[Activity diagram: P2P network data download]{
        Diagrama de Atividades: Solicitação de Dados pela Rede P2P.\\
        Fonte: os autores.
    }
    \label{fig:Solicitacao_Dados}
\end{figure}

O processo começa com a etapa onde o usuário indica qual o dado deseja acessar; Exemplo:\textit{board}, \textit{thread}, comentário (Todos exemplos de postagens do \textit{4chan}). Depois a rede procura pela \textit{hash} desse \textit{post} na rede para descobrir qual nodo tem esse dado, verificando-se existem nodos da lista de nodos confiáveis do usuário, não tendo nodos da lista confiáveis oferece a opção de baixar
de um nodo não confiável. Após essa primeira etapa é realizado a comunicação entre os nodos e transferido os arquivos, por fim o usuário escolhe se deseja manter salvo os arquivos 
em usa máquina. 
\section{Diagramas de Sequência}
Nesta seção serão apresentados os diagramas de sequencia das ferramentas quem tem contato direto com o usuário(\textit{Scraper} e Rede \textit{Peer-to-Peer}), com foco na comunicação entre os objetos e nos tempos de execução e reposta das funcionalidade destas ferramentas.
\subsection{Extração de Dados}
Esta seção faz a analise dos tempos de execução e resposta do processo de extração de dados descrito na seção~\ref{subsec:extracao_de_dados}.
A representação dos tempos de execução podem ser vistas através do diagrama de sequencia representado na figura~\ref{fig:seq_extracao_dados}.

\begin{figure}[htb]
    \includegraphics[width=\textwidth]{fig/DiagramadeSequenciaScraper.pdf}
    \caption[Diagrama de Sequência: Extração de Dados]{
        Diagrama de Sequência: Extração de Dados.\\
        Fonte: os autores.
    }
    \label{fig:seq_extracao_dados}
\end{figure}

O processo inicia com a solicitação síncrona do usuário ao \textit{Scraper}, que envia uma requisição síncrona para a API do \textit{4chan}, o \textit{Scraper} fica a espera de uma resposta assíncrona do \textit{4chan} e envia uma resposta assíncrona para o usuário. 
O tempo de espera do usuário deve ser perto do tempo de espera de uma aplicação da web 2.0, visto que o processo é o mesmo. A única diferença de tempo que pode ser esperada depende do formato do dado que o usuário escolheu
se o formato for diferente da resposta da API um tempo de formatação do \textit{Scraper} deverá ser adicionado ao tempo de espera esperado.

\subsection{Solicitação de Dados pela Rede P2P}
Esta seção faz a analise dos tempos de execução e resposta do processo de solicitação de dados pela \textit{rede peer-to-peer} descrito na seção~\ref{subsec:Solicitacao_Dados}. A representação dos tempos de execução podem ser vistas através do diagrama de sequencia representado na figura~\ref{fig:seq_Solicitacao_dados}. 

\begin{figure}[H]
    \includegraphics[width=\textwidth]{fig/diagrama-sequencia-rede-p2p.pdf}
    \caption[Sequence diagram: requesting data to peers]{
        Diagrama de Sequência: Solicitação de Dados pela Rede P2P.\\
        Fonte: os autores.
    }
    \label{fig:seq_Solicitacao_dados}
\end{figure}

O processo inicia com solicitação síncrona do usuário a rede \textit{peer-to-peer}, indicando os dados
requisitados, a rede \textit{peer-to-peer} envia uma resposta assíncrona que pode ser dado encontrado ou dado não encontrado,
a partir desse dado o usuário envia uma solicitação ao \textit{peer} que contém o conteúdo desejado
e começa o download do conteúdo. Este processo contém algumas variáveis, como a conexão entre os dois \textit{peers}
e quantos \textit{peers} na rede contém aquele dado, pois na rede \textit{peer-to-peer} é possível baixar partes de um dado de fontes diferentes. Assim o tempo de execução pode ser menor ou maior que o da plataforma da web 2.0.
\section{Diagramas de Casos de Uso}
Nesta seção serão apresentados os diagramas de caso de uso e os atores
modelados por ferramenta. O objetivo do uso desta modelagem é abordar mais profundamente as funcionalidades das ferramentas atráves da modelagem de casos de uso. 
\subsection{Atores}
Por o projeto ser um conjunto de ferramentas, elas serão utilizadas apenas por um tipo de Ator, o usuário. Ele será responsável por fornecer os dados de entrada para as ferramentas, e as configurações que melhor desejar e que possam ser fornecidas as ferramentas.
\begin{figure}[H]
    \centering
    \includesvg[width=0.07\textwidth]{fig/Ator.svg}
    \caption[Diagrama do Ator]{\label{fig:diagrama_do_ator}
        Diagrama do Ator\\
        Fonte: os autores.
    }
\end{figure}
\subsection{Casos de Uso Scraper}
Nesta seção será apresentado os casos de uso para o ator usuário na ferramenta \textit{Scraper}.
\subsubsection{UC01 - Especificar o Formato dos Dados}
Permite ao usuário especificar ao \textit{Scraper} qual o formato dos dados que ele espera receber. Para isso o usuário deve passar como parâmetro para o \textit{Scraper} essa informação, ou configurá-lo previamente com o formato favorito.
\subsubsection{UC02 - Baixar conteúdo de Texto de uma Thread do 4chan}
Permite ao usuário baixar todo o texto de uma thread do 4chan. Para isso o usuário deve passar como parâmetro para o \textit{Scraper} a id da thread que deseja baixar o texto.
\subsubsection{UC03 - Baixar conteúdo de mídia(fotos, vídeos)  de uma Thread do 4chan}
Permite ao usuário baixar todo a mídia de uma thread do 4chan. Para isso o usuário deve passar como parâmetro para o \textit{Scraper} a id da thread que deseja baixar a mídia.
\subsubsection{UC04 - Baixar todo o conteúdo de uma Thread do 4chan}
Permite ao usuário baixar todo a conteúdo de uma thread do 4chan (mídia e texto). Para isso o usuário deve passar como parâmetro para o \textit{Scraper} a id da thread que deseja baixar todo o conteúdo.
\begin{figure}[H]
    \centering
    \includesvg[width=0.5\textwidth]{fig/Ator_Scraper1.svg}
    \caption[Casos de uso Scraper UC01, UC02, UC03, e UC04]{\label{fig:Ator_Scraper1}
        Casos de uso Scraper UC01,UC02,UC03,UC04\\
        Fonte: os autores.
    }
\end{figure}
\subsubsection{UC05 - Especificar Threads para monitoramento}
Permite ao usuário informar uma thread para monitoramento no 4chan, assim o scraper fará consultas constantemente, buscando e baixando alterações na thread. Para isso o usuário deve passar como parâmetro para o \textit{Scraper} a id da thread que deseja monitorar.  
\subsubsection{UC06 - Especificar Boards para monitoramento}
Permite ao usuário informar um board para monitoramento no 4chan, assim o scraper fará consultas constantemente, buscando e baixando alterações no board. Para isso o usuário deve passar como parâmetro para o \textit{Scraper} a id do board que deseja monitorar.  
\subsubsection{UC07 - Interromper o monitoramento de uma Thread que estiver fechada}
Permite ao usuário configurar o \textit{Scraper} para parar de monitorar threads que estiverem fechadas.

\begin{figure}[H]
    \centering
    \includesvg[width=0.65\textwidth]{fig/Ator_Scraper2.svg}
    \caption[Casos de uso Scraper UC05,UC06,UC07]{\label{fig:Ator_Scraper2}
        Casos de uso Scraper UC05,UC06,UC07\\
        Fonte: os autores.
    }
\end{figure}

\subsection{Casos de Uso Networking}

Nesta seção será apresentado os casos de uso para o ator usuário com ferramenta \textit{Networking}.
\subsubsection{UC01 - Procurar Conteúdo pela Rede}
Permite ao usuário procurar conteúdo pela rede peer-to-peer, para isso ele deve passar o nome ou hash do arquivo que deseja.
\subsubsection{UC02 - Requisitar Conteúdo pela Rede}
Permite ao usuário requisitar conteúdo pela rede peer-to-peer, para isso ele deve passar o nome ou hash do arquivo que deseja.
\subsubsection{UC03 - Fornecer Conteúdo pela Rede}
Permite ao usuário fornecer conteúdo pela rede peer-to-peer, para isso ele deve ter o arquivo e configurar a rede para fornecer o conteúdo.

\begin{figure}[H]
    \centering
    \includesvg[width=0.5\textwidth]{fig/Ator_Networking1.svg}
    \caption[Casos de uso Networking UC01,UC02,UC03]{\label{fig:Ator_Networking2}
        Casos de uso Networking UC01,UC02,UC03\\
        Fonte: os autores.
    }
\end{figure}

\subsection{Casos de Uso Criptografia}

Nesta seção será apresentado os casos de uso para o ator usuário com ferramenta \textit{Criptografia}.
\subsubsection{UC01 - Criar um par de chaves criptográficas}
Permite ao usuário criar um par de chaves criptográficas para garantir a autenticidade e integridade no compartilhamento de arquivos na rede.
\subsubsection{UC02 - Criar lista de nodos confiáveis}
Permite ao usuário criar uma lista de nodos confiáveis, para isso o usuário deve armazenar uma lista com a chave pública dos nodos que julgar confiáveis.
\subsubsection{UC03 - Exportar minha chave pública para Compartilhamento}
Permite ao usuário exportar a chave pública no formato cleartext para compartilhamento. Para isso o usuário necessita ter um par de chaves criptográficas. 
\subsubsection{UC04 - Assinar arquivos com a minha chave privada}
Permite ao usuário assinar arquivos com sua chave privada para que os outros peers da rede que o mantém como nodo confiável e tem sua chave publica possam reconhecer a assinatura e identificar que ele é um peer confiável. Para isso o usuário deve ter um par de chaves criptográficas.
\subsubsection{UC05 - Verificar se a assinatura de um arquivo pertence a minha lista de nodos confiáveis}
Permite ao usuário identificar a assinatura de um nodo confiável a partir da chave publica na lista de nodos confiáveis. Para isso o usuário necessita ter armazenado na sua lista de nodos confiáveis a chave publica do nodo.

\begin{figure}[htb]
    \centering
    \includegraphics[width=\textwidth]{fig/use-case-diagram-crypto.pdf}
    \caption[Use case diagram: cryptography module]{
        Diagram de casos de uso: módulo de criptografia.\\
        Fonte: os autores.
    }
\end{figure}

\section{Detalhamento dos Casos de Uso}

\subsection{Especificar o Formato dos Dados}

\subsubsection{Descrição}
Permite ao usuário especificar ao \textit{Scraper} qual o formato dos dados que ele espera receber. Para isso o usuário deve passar como parâmetro para o \textit{Scraper} essa informação, ou configurá-lo previamente com o formato favorito.
\subsubsection{Atores : Usuário}
\subsubsection{Pré-Condição}
O usuário deve informar o formato dos dados.
\subsubsection{Pós-Condição}
O usuário tem o dados no formato desejado.
\subsubsection{Fluxo Principal}
\begin{enumerate}
    \item Abre o Scraper pela linha de comando;
    \item Passa como parâmetro o id do dado que deseja e o formato dos dados;
    \item Scraper realiza a busca do dado;
    \item Scraper realiza a formatação do dado;
    \item O usuário abre o dado;
\end{enumerate}
\subsubsection{Fluxo Alternativo}
\begin{itemize}
    \item Dado não existe no 4chan: ocorre quando no passo 3 scraper não acha o dado indicado pelo usuário no 4chan.
    \begin{enumerate}
        \item Mostra mensagem de erro;
        \item Retorna ao passo 2.
    \end{enumerate}
    \item Scraper não reconhece o formato especificado: ocorre quando no passo 4 o scraper vai formatar o dado indicado pelo usuário e não consegue identificar o formato desejado.
    \begin{enumerate}
        \item Mostra mensagem de erro;
        \item Retorna ao passo 2.
    \end{enumerate}
\end{itemize}


\subsection{Baixar conteúdo de Texto de uma Thread do 4chan}
\subsubsection{Descrição}
Permite ao usuário baixar todo o texto de uma thread do 4chan. Para isso o usuário deve passar como parâmetro para o \textit{Scraper} a id da thread que deseja baixar o texto.
\subsubsection{Atores : Usuário}
\subsubsection{Pré-Condição}
O usuário deve informar o id das threads que deseja salvar.
\subsubsection{Pós-Condição}
O usuário terá o todo conteúdo de texto da thread indicada.
\subsubsection{Fluxo Principal}
\begin{enumerate}
    \item Abre o Scraper pela linha de comando;
    \item Passa como parâmetro o id do dado que deseja e o formato dos dados;
    \item Scraper realiza a busca do dado;
    \item Scraper realiza a formatação do dado;
    \item O usuário abre o dado;
\end{enumerate}
\subsubsection{Fluxo Alternativo}
\begin{itemize}
    \item Dado não existe no 4chan: ocorre quando no passo 3 scraper não acha o dado indicado pelo usuário no 4chan.
    \begin{enumerate}
        \item Mostra mensagem de erro;
        \item Retorna ao passo 2.
    \end{enumerate}
    \item Scraper não reconhece o formato especificado: ocorre quando no passo 4 o scraper vai formatar o dado indicado pelo usuário e não consegue identificar o formato desejado.
    \begin{enumerate}
        \item Mostra mensagem de erro;
        \item Retorna ao passo 2.
    \end{enumerate}
\end{itemize}

\subsection{Baixar conteúdo de mídia de uma Thread do 4chan}
\subsubsection{Descrição}
Permite ao usuário baixar todo a mídia de uma thread do 4chan. Para isso o usuário deve passar como parâmetro para o \textit{Scraper} a id da thread que deseja baixar a mídia.
\subsubsection{Atores : Usuário}
\subsubsection{Pré-Condição}
O usuário deve informar o id das threads que deseja salvar.
\subsubsection{Pós-Condição}
O usuário terá o todo conteúdo de mídia(fotos,vídeos) da thread indicada.
\subsubsection{Fluxo Principal}
\begin{enumerate}
    \item Abre o Scraper pela linha de comando;
    \item Passa como parâmetro o id do dado que deseja e o formato dos dados;
    \item Scraper realiza a busca do dado;
    \item Scraper realiza a formatação do dado;
    \item O usuário abre o dado;
\end{enumerate}
\subsubsection{Fluxo Alternativo}
\begin{itemize}
    \item Dado não existe no 4chan: ocorre quando no passo 3 scraper não acha o dado indicado pelo usuário no 4chan.
    \begin{enumerate}
        \item Mostra mensagem de erro;
        \item Retorna ao passo 2.
    \end{enumerate}
    \item Scraper não reconhece o formato especificado: ocorre quando no passo 4 o scraper vai formatar o dado indicado pelo usuário e não consegue identificar o formato desejado.
    \begin{enumerate}
        \item Mostra mensagem de erro;
        \item Retorna ao passo 2.
    \end{enumerate}
\end{itemize}

\subsection{Baixar todo o conteúdo de uma Thread do 4chan}
\subsubsection{Descrição}
Permite ao usuário baixar todo a conteúdo de uma thread do 4chan (mídia e texto). Para isso o usuário deve passar como parâmetro para o \textit{Scraper} a id da thread que deseja baixar todo o conteúdo.
\subsubsection{Atores : Usuário}
\subsubsection{Pré-Condição}
O usuário deve informar o id das threads que deseja salvar.
\subsubsection{Pós-Condição}
O usuário terá o todo conteúdo da thread indicada.
\subsubsection{Fluxo Principal}
\begin{enumerate}
    \item Abre o Scraper pela linha de comando;
    \item Passa como parâmetro o id do dado que deseja e o formato dos dados;
    \item Scraper realiza a busca do dado;
    \item Scraper realiza a formatação do dado;
    \item O usuário abre o dado;
\end{enumerate}
\subsubsection{Fluxo Alternativo}
\begin{itemize}
    \item Dado não existe no 4chan: ocorre quando no passo 3 scraper não acha o dado indicado pelo usuário no 4chan.
    \begin{enumerate}
        \item Mostra mensagem de erro;
        \item Retorna ao passo 2.
    \end{enumerate}
    \item Scraper não reconhece o formato especificado: ocorre quando no passo 4 o scraper vai formatar o dado indicado pelo usuário e não consegue identificar o formato desejado.
    \begin{enumerate}
        \item Mostra mensagem de erro;
        \item Retorna ao passo 2.
    \end{enumerate}
\end{itemize}


\subsection{Especificar Threads para monitoramento}
\subsubsection{Descrição}
Permite ao usuário informar uma thread para monitoramento no 4chan, assim o scraper fará consultas constantemente, buscando e baixando alterações na thread. Para isso o usuário deve passar como parâmetro para o \textit{Scraper} a id da thread que deseja monitorar. 
\subsubsection{Atores : Usuário}
\subsubsection{Pré-Condição}
O usuário deve informar o id das threads que deseja monitorar.
\subsubsection{Pós-Condição}
O usuário terá o conteúdo da thread constantemente atualizado.
\subsubsection{Fluxo Principal}
\begin{enumerate}
    \item Abre o Scraper pela linha de comando;
    \item Passa como parâmetro o id da thread que deseja monitorar;
    \item Scraper realiza a busca da thread;
    \item Scraper começa o monitoramento da thread;
\end{enumerate}
\subsubsection{Fluxo Alternativo}
\begin{itemize}
    \item Thread não existe no 4chan: ocorre quando no passo 3 scraper não acha a thread indicado pelo usuário no 4chan.
    \begin{enumerate}
        \item Mostra mensagem de erro;
        \item Retorna ao passo 2.
    \end{enumerate}
\end{itemize}


\subsection{Especificar Boards para monitoramento}
\subsubsection{Descrição}
Permite ao usuário informar um board para monitoramento no 4chan, assim o scraper fará consultas constantemente, buscando e baixando alterações no board. Para isso o usuário deve passar como parâmetro para o \textit{Scraper} a id do board que deseja monitorar.  
\subsubsection{Atores : Usuário}
\subsubsection{Pré-Condição}
O usuário deve informar o id das boards que deseja monitorar.
\subsubsection{Pós-Condição}
O usuário terá o conteúdo do board constantemente atualizado.
\subsubsection{Fluxo Principal}
\begin{enumerate}
    \item Abre o Scraper pela linha de comando;
    \item Passa como parâmetro o id do board que deseja monitorar;
    \item Scraper realiza a busca do board;
    \item Scraper começa o monitoramento do board;
\end{enumerate}
\subsubsection{Fluxo Alternativo}
\begin{itemize}
    \item Board não existe no 4chan: ocorre quando no passo 3 scraper não acha o board indicado pelo usuário no 4chan.
    \begin{enumerate}
        \item Mostra mensagem de erro;
        \item Retorna ao passo 2.
    \end{enumerate}
\end{itemize}


\subsection{Interromper o monitoramento de uma Thread que estiver fechada}
\subsubsection{Descrição}
Permite ao usuário configurar o \textit{Scraper} para parar de monitorar threads que estiverem fechadas.
\subsubsection{Atores : Usuário}
\subsubsection{Pré-Condição}
Nenhuma pré-condição é necessária.
\subsubsection{Pós-Condição}
O scraper deixará de buscar atualizações na thread.
\subsubsection{Fluxo Principal}
\begin{enumerate}
    \item Abre o Scraper pela linha de comando;
    \item Passa como parâmetro o id da thread;
    \item Scraper adiciona ao monitoramento da thread a flag de fechada ou aberto;
\end{enumerate}
\subsubsection{Fluxo Alternativo}
\begin{itemize}
    \item Thread não existe no 4chan: ocorre quando no passo 3 scraper não acha a Thread indicado pelo usuário no 4chan.
    \begin{enumerate}
        \item Mostra mensagem de erro;
        \item Retorna ao passo 2.
    \end{enumerate}
    \item Scraper não monitora essa thread: ocorre quando no passo 3 scraper não tem essa Thread na sua lista de monitoração.
    \begin{enumerate}
        \item Mostra mensagem de erro;
        \item Retorna ao passo 2.
    \end{enumerate}
\end{itemize}


\subsection{Procurar Conteúdo pela Rede}
\subsubsection{Descrição}
Permite ao usuário procurar conteúdo pela rede peer-to-peer, para isso ele deve passar o nome ou hash do arquivo que deseja.
\subsubsection{Atores : Usuário}
\subsubsection{Pré-Condição}
O usuário deve informar o nome ou a hash do arquivo que deseja buscar.
\subsubsection{Pós-Condição}
O usuário saberá se um dado existe na rede, e os nodos que contem aquele dado.
\subsubsection{Fluxo Principal}
\begin{enumerate}
    \item Se conecta a rede pela linha de comando;
    \item Passa como parâmetro o nome ou hash do arquivo desejado;
    \item A rede pesquisa o arquivo pela rede;
    \item A rede informa uma lista de usuário que contém o arquivo;
\end{enumerate}
\subsubsection{Fluxo Alternativo}
\begin{itemize}
    \item O arquivo não existe na rede: ocorre quando no passo 3 a rede não acha o arquivo indicado pelo usuário.
    \begin{enumerate}
        \item Mostra mensagem de erro;
        \item Retorna ao passo 2.
    \end{enumerate}
\end{itemize}


\subsection{Requisitar Conteúdo pela Rede}
\subsubsection{Descrição}
Permite ao usuário requisitar conteúdo pela rede peer-to-peer, para isso ele deve passar o nome ou hash do arquivo que deseja.
\subsubsection{Atores : Usuário}
\subsubsection{Pré-Condição}
O usuário deve informar o nome ou a hash do arquivo que deseja requisitar.
\subsubsection{Pós-Condição}
O usuário terá o arquivo requisitado.
\subsubsection{Fluxo Principal}
\begin{enumerate}
    \item Se conecta a rede pela linha de comando;
    \item Passa como parâmetro o nome ou hash do arquivo desejado;
    \item A rede pesquisa o arquivo pela rede;
    \item A rede informa uma lista de usuário que contém o arquivo;
    \item O usuário informa se conecta a um dos usuários da lista;
    \item O usuário baixa o arquivo desejado;
\end{enumerate}
\subsubsection{Fluxo Alternativo}
\begin{itemize}
    \item O arquivo não existe na rede: ocorre quando no passo 3 a rede não acha o arquivo indicado pelo usuário.
    \begin{enumerate}
        \item Mostra mensagem de erro;
        \item Retorna ao passo 2.
    \end{enumerate}
    \item Não existe peers Confiáveis: ocorre quando no passo 4 nenhum dos peers está na lista de confiáveis do usuário.
    \begin{enumerate}
        \item Mostra mensagem se o usuário deseja prosseguir com um peer não confiável;
        \item Segue ao passo 5 ou encerra a execução.
    \end{enumerate}
\end{itemize}


\subsection{Fornecer Conteúdo pela Rede}

\subsubsection{Descrição}

Permite ao usuário fornecer conteúdo pela rede peer-to-peer, para isso ele deve ter o arquivo e configurar a rede para fornecer o conteúdo.

\subsubsection{Atores : Usuário}

\subsubsection{Pré-Condição}

O usuário deve ter armazenado os arquivos que deseja fornecer.

\subsubsection{Pós-Condição}

O usuário estará disponível na para outros usuários quando pesquisarem pelo arquivo servido. 

\subsubsection{Fluxo Principal}

\begin{enumerate}
    \item Se conecta a rede pela linha de comando;
    \item Passa como parâmetro o local do arquivo em sua maquina;
    \item A rede lê o arquivo;
    \item A rede gera uma hash e espoem o arquivo para a rede;
\end{enumerate}

\subsubsection{Fluxo Alternativo}

\begin{itemize}
    \item O arquivo não existe no caminho informado: ocorre quando no passo 3 a rede não acha o arquivo indicado pelo usuário.
    \begin{enumerate}
        \item Mostra mensagem de erro;
        \item Retorna ao passo 2.
    \end{enumerate}
\end{itemize}


\subsection{Criar um par de chaves criptográficas}
\subsubsection{Descrição}
Permite ao usuário criar um par de chaves criptográficas para garantir a autenticidade e integridade no compartilhamento de arquivos na rede.
\subsubsection{Atores : Usuário}
\subsubsection{Pré-Condição}
Nenhuma pré-condição é necessária.
\subsubsection{Pós-Condição}
O usuário terá um chave pública e uma chave criptográfica.
\subsubsection{Fluxo Principal}
\begin{enumerate}
    \item Se conecta ao modulo de criptografia pela linha de comando;
    \item Executa o método para criação de chaves criptográficas;
    \item O modulo de criptografia gera um par de chaves criptográficas para o usuário;
\end{enumerate}


\subsection{Criar lista de nodos confiáveis}
\subsubsection{Descrição}
Permite ao usuário criar uma lista de nodos confiáveis, para isso o usuário deve armazenar uma lista com a chave pública dos nodos que julgar confiáveis.
\subsubsection{Atores : Usuário}
\subsubsection{Pré-Condição}
O usuário deve ter no mínimo uma chave publica de um nodo para criar a lista.
\subsubsection{Pós-Condição}
O usuário terá uma lista de nodos confiáveis que poderá ser usado para verificar a identidade de um usuário e auxiliar no download de dados pela rede.
\subsubsection{Fluxo Principal}
\begin{enumerate}
    \item Se conecta a rede pela linha de comando;
    \item Executa o método para criação de lista de nodos confiáveis passando como parâmetro a chave publica de um nodo confiável;
    \item A rede gera a lista de nodos confiáveis;
\end{enumerate}
\subsubsection{Fluxo Alternativo}
\begin{itemize}
    \item A lista já foi criada: ocorre quando no item 3 já ter uma lista criada.
    \begin{enumerate}
        \item Mostra mensagem de erro, perguntado se deseja substituir a lista;
        \item Substitui a lista ou encerra o programa.
    \end{enumerate}
\end{itemize}


\subsection{Exportar minha chave pública para Compartilhamento}
\subsubsection{Descrição}
Permite ao usuário exportar a chave pública no formato cleartext para compartilhamento. Para isso o usuário necessita ter um par de chaves criptográficas. 
\subsubsection{Atores : Usuário}
\subsubsection{Pré-Condição}
O usuário precisa ter chaves criptográficas.
\subsubsection{Pós-Condição}
O usuário terá sua chave publica num formato de fácil compartilhamento.
\subsubsection{Fluxo Principal}
\begin{enumerate}
    \item Se conecta a rede pela linha de comando;
    \item Executa o método para exportar a chave publica no formato cleartext;
    \item A rede gera a converte a chave publica para cleartext e informa ao usuário;
\end{enumerate}
\subsubsection{Fluxo Alternativo}
\begin{itemize}
    \item O usuário não tem chave publica cadastrada: ocorre quando no item 3 o usuário não tem um par de chaves criptográficas criadas.
    \begin{enumerate}
        \item Mostra mensagem de erro;
        \item Encerra o programa.
    \end{enumerate}
\end{itemize}


\subsection{Assinar arquivos com a minha chave privada}
\subsubsection{Descrição}
Permite ao usuário assinar arquivos com sua chave privada para que os outros peers da rede que o mantém como nodo confiável e tem sua chave publica possam reconhecer a assinatura e identificar que ele é um peer confiável. Para isso o usuário deve ter um par de chaves criptográficas.
\subsubsection{Atores : Usuário}
\subsubsection{Pré-Condição}
O usuário precisa ter chaves criptográficas.
\subsubsection{Pós-Condição}
O arquivo poderá ser verificado por usuários que tenham a chave pública correspondente.
\subsubsection{Fluxo Principal}
\begin{enumerate}
    \item Se conecta a rede pela linha de comando;
    \item Executa o método para assinar um arquivo passando como parâmetro o caminho do arquivo na maquina do usuário;
    \item A rede gera uma assinatura no arquivo;
\end{enumerate}
\subsubsection{Fluxo Alternativo}
\begin{itemize}
    \item O usuário não tem chave publica cadastrada: ocorre quando no item 3 o usuário não tem um par de chaves criptográficas criadas.
    \begin{enumerate}
        \item Mostra mensagem de erro;
        \item Encerra o programa.
    \end{enumerate}
     \item O arquivo não existe: ocorre quando no item 3 o programa não acha o arquivo no caminho especificado pelo usuário.
    \begin{enumerate}
        \item Mostra mensagem de erro;
        \item Retorna ao passo 2.
    \end{enumerate}
\end{itemize}

\subsection{Verificar se a assinatura de um arquivo pertence a minha lista de nodos confiáveis}
\subsubsection{Descrição}
Permite ao usuário identificar a assinatura de um nodo confiável a partir da chave publica na lista de nodos confiáveis. Para isso o usuário necessita ter armazenado na sua lista de nodos confiáveis a chave publica do nodo.
\subsubsection{Atores : Usuário}
\subsubsection{Pré-Condição}
O usuário precisa ter uma lista de nodos confiáveis.
\subsubsection{Pós-Condição}
O usuário terá verificado a identidade de quem assinou aquele arquivo.
\subsubsection{Fluxo Principal}
\begin{enumerate}
    \item Se conecta a rede pela linha de comando;
    \item Executa o método para buscar um arquivo na passando como parâmetro o nome ou a hash do arquivo;
    \item A rede pesquisa o arquivo pela rede;
    \item A rede informa uma lista de usuário que contém o arquivo;
    \item O usuário seleciona o arquivo e usuário e pede para rede verificar a assinatura;
    \item A rede verifica a assinatura.
\end{enumerate}
\subsubsection{Fluxo Alternativo}
\begin{itemize}
    \item O arquivo não existe na rede: ocorre quando no passo 3 a rede não acha o arquivo indicado pelo usuário.
    \begin{enumerate}
        \item Mostra mensagem de erro;
        \item Retorna ao passo 2.
    \end{enumerate}
\end{itemize}


\chapter{Metodologia}

A metodologia empregada para o planejamento e desenvolvimento do sistema deste trabalho de conclusão é o tradicional modelo cascata.
Esse modelo funciona como um exemplo de um processo dirigido a planos, onde, em princípio deve-se planejar e programar todas as atividades do processo até a implementação e a manutenção do dele~\cite{SOMMERVILLE1}.

Consiste numa sequência de fases em que cada fase dá suporte para a próxima, assim o produto de cada fase torna-se a entrada para a seguinte~\cite{BALTZAN1}.
A fase inicial é definição de requisitos que busca estabelecer os serviços, restrições e metas do sistema, normalmente por meio de consultas com o usuário~\cite{SOMMERVILLE1}; a próxima fase projeto de sistema e software é a fase em que a identificação e descrição das abstrações fundamentais do sistema é realizada, alocando os requisitos do sistema por meio de uma arquitetura geral do sistema~\cite{SOMMERVILLE1}; a terceira fase implementação e teste unitário é a fase de desenvolvimento do projeto como um conjunto de programas ou unidades de programa, utilizando o teste unitário para verificar de cada unidade, tendo certeza de que a mesma cumpra sua especificação~\cite{SOMMERVILLE1}; a quarta fase integração e teste de sistema é quando as unidades criadas na fase anterior de maneira independente são integradas e testadas como um sistema único e completo, verificando se todos os requisitos estão sendo atendidos~\cite{SOMMERVILLE1}.
Na fase final, Operação e manutenção, é o momento em que o sistema é de fato instalado e colocado em uso, é onde os eventuais erros que não apareceram anteriormente são resolvidos e, também, onde novos requisitos podem ser descobertos~\cite{SOMMERVILLE1}.

\begin{figure}[H]
    \centering
    \includegraphics[width=\textwidth]{fig/waterfall-development-model.pdf}
    \caption[Waterfall development model]{\label{fig:Modelo_Cascata}
        Diagrama ilustrando etapas e fluxos do Modelo Cascata de desenvolvimento e gerenciamento de projetos.\\
        Fonte: SOMMERVILLE~\cite{SOMMERVILLE1}.
    }
\end{figure}

Por mais que essa seja uma metodologia não mais tão utilizada pelas empresas devido à sua característica de precisão extrema que não leva em consideração mudanças durante o desenvolvimento do projeto~\cite{BALTZAN1}, para este trabalho ela se torna uma boa escolha, pois se trata de um Trabalho de Conclusão de Curso, com um único desenvolvedor e prazos muito bem definidos.
Devido à inexistência de um cliente e visando um público alvo, reuniões com o cliente são impossíveis e testes durante o desenvolvimento do trabalho com alguém que faça parte do público alvo não acontecerão facilmente, o modelo cascata se torna a melhor opção como metodologia de desenvolvimento do projeto.
A partir de agora as fases alvo são: a codificação, os testes e a implementação/manutenção do sistema, concluindo assim o ciclo completo do modelo tradicional em cascata.

%\chapter{Cronograma}

%Apresentamos a seguir um cronograma relatando nossas atividades desenvolvidas e nosso planejamento, relatando as atividades que já foram e que serão desenvolvidas.

%\begin{enumerate}
%    \item \label{cron:search} Seleção e análise das redes sociais atuais.
%        \subitem Mapear vantagens e desvantagens delas.
%    \item \label{cron:background} Consolidação da fundamentação teórica.
%    \item \label{cron:architecture} Elaboração da proposta de Arquitetura.
%    \item \label{cron:proposal} Entrega da proposta de TC.
%    \item \label{cron:requirements} Identificação dos requisitos do sistema.
 %   \item \label{cron:diagrams} Elaboração da modelagem do sistema.
%    \item \label{cron:methodology} Definição da metodologia de desenvolvimento.
%    \item \label{cron:esc-tcI} Entrega do volume final de TC1.
%    \item \label{cron:scraper} Desenvolvimento do \textit{Scraper}.
%    \item \label{cron:val1}  Teste e validação do \textit{Scraper} e estrutura de dados.
%    \item \label{cron:network} Desenvolvimento da rede.
%    \item \label{cron:val2} Teste e validação da rede e do protocolo da rede.
%    \item \label{cron:integration} Integração entre \textit{Scraper} e rede de dados.
%    \item \label{cron:val3} Testes da Integração.
%    \item \label{cron:gui} Desenvolvimento da interface gráfica.
%    \item \label{cron:val4} Testes e validações finais do sistema.
 %   \item \label{cron:poster} Entrega dos Cartazes.
%    \item \label{cron:esc-tcII} Entrega da volume final de TC2.
%\end{enumerate}

%\definecolor{midgray}{gray}{.5}
%\begin{table}[!htbp]
%    \centering
%    \begin{tabular}{|c|c|c|c|c|c|}
%        \hline
%                                & %\multicolumn{5}{c|}{2021}                             %                                                            \\
%        \hline
%                                & MAR                       & ABR                 & MAI                 & JUN                 & JUL                 \\
 %       \hline
 %      \ref{cron:search}       & \cellcolor{midgray}       &                     &                     &                     &                     \\
 %       \hline
 %       \ref{cron:background}   & \cellcolor{midgray}       &                     &                     &                     &                     \\
 %       \hline
 %       \ref{cron:architecture} &                           & \cellcolor{midgray} &                     &                     &                     \\
%        \hline
%        \ref{cron:proposal}     &                           & \cellcolor{midgray} &                     &                     &                     \\
%        \hline
%        \ref{cron:requirements} &                           & \cellcolor{midgray} &                     &                     &                     \\
%        \hline
%        \ref{cron:diagrams}     &                           & \cellcolor{midgray} &                     &                     &                     \\
%        \hline
%        \ref{cron:methodology}  &                           & \cellcolor{midgray} &                     &                     &                     \\
%        \hline
%        \ref{cron:esc-tcI}      &                           & \cellcolor{midgray} &                     &                     &                     \\
%        \hline
 %       \ref{cron:scraper}      &                           &                     & \cellcolor{midgray} &                     &                     \\
%        \hline
 %       \ref{cron:val1}         &                           &                     & \cellcolor{midgray} &                     &                     \\
%        \hline
%        \ref{cron:network}      &                           &                     & \cellcolor{midgray} &                     &                     \\
%        \hline
%        \ref{cron:val2}         &                           &                     & \cellcolor{midgray} &                     &                     \\
 %       \hline
%        \ref{cron:integration}  &                           &                     &                     & \cellcolor{midgray} &                     \\
%        \hline
%        \ref{cron:val3}         &                           &                     &                     & \cellcolor{midgray} &                     \\
%        \hline
%        \ref{cron:gui}          &                           &                     &                     & \cellcolor{midgray} &                     \\
%        \hline
%        \ref{cron:val4}         &                           &                     &                     & \cellcolor{midgray} &                     \\
%        \hline
%        \ref{cron:poster}       &                           &                     &                     & \cellcolor{midgray} &                     \\
 %       \hline
%        \ref{cron:esc-tcII}     &                           &                     &                     &                     & \cellcolor{midgray} \\
%        \hline
%    \end{tabular}
%    \caption{\label{tab:schedule}
%        Cronograma de atividades.\\
%        Fonte: os autores.
%    }
%\end{table}

\chapter{Implementação}

Esse capitulo esclarece o que foi implementado e o que não foi implementado referente ao que foi proposto, bem como implementações não previstas que foram necessárias para dar suporte as soluções propostas.

\section{Implementados}

Nessa seção será descrito o que foi implementado conforme previsto na proposta, divido pelas ferramentas.

\subsection{Archive-Chan}

O Archive-chan ou módulo scraper foi totalmente implementado e seus requisitos cumpridos.
É possivel baixar threads e boards do 4chan, bem como escolher se quer baixar somente arquivos de texto ou arquivos de mídia também.
O projeto foi totalmente feito em Python e se encontra no repositório do Github \url{https://github.com/cardoso-neto/archive-chan}.
No repositório também se encontra um tutorial para a instalação e uso do Archive-chan.

\subsection{Network-Chan}

O Network-chan ou módulo de rede peer-to-peer foi totalmente implementada e seus requisitos cumpridos.
É possível procurar, servir, e baixar arquivos utilizando o Network-Chan.
Para realizar esse feito foi utilizado a bilbioteca em Python ipfshttpclient, graças a ela podemos criar um client IPFS e nos conectarmos ao IPFS, permitindo assim adicionar e requisitar arquivos conectando os peers pelo IPFS.
A ferramenta foi totalmente feita em Python e está disponível no seguinte link do Github \url{https://github.com/cardoso-neto/network-chan}.
No repositório é descrito como baixar a ferramenta e relatado a dependência com o IPFS.

\subsection{Crypto-Chan}

O módulo de criptográfia (Crypto-chan) está em desenvolvimento.
Sendo mais tecnicamente uma ferramenta auxiliar do modulo de rede peer-to-peer Network-Chan que será usada opcionalmente após o final de cada download para verificar se alguém em que confiemos assinou aquela hash.

\section{Implementações Não Planejadas}

Nessa seção será descrito o que foi implementado e não constava na proposta.

\subsection{Chan-url-lib}

Implementamos um biblioteca que, com o auxílio de algumas expressões regulares que escolhemos, canonicaliza URLs de threads de chans em URIs estilo \textit{paths} UNIX para que possamos montar nossa estrutura de pastas e usar essas strings como chaves na DHT.
Por exemplo, \url{https://boards.4chan.org/pol/thread/327698754/pol-humor-thread} ou \url{https://boards.4chan.org/pol/thread/327698754#p327726735} serializam como \url{4chan.org/pol/327698754}.

Disponível em \url{https://github.com/cardoso-neto/chan-url-lib} junto com seus testes unitários.

\subsection{Renderizador de threads}

Baixar o conteúdo do 4chan se mostrou insuficiente, sem uma maneira pronta (out of the box) de interagir com os dados.
Então desenvolvemos, utilizando HTML templating, um layout muito similar ao 4chan (para preservar o formato das discussões) inteiramente offline para que os usuários possam navegar nas threads que escolherem preservar.

Por estarmos apressados, desenvolvemos essa funcionalidade, que deveria ser um programa separado, acoplada ao archive-chan\footnote{disponível em \url{https://github.com/cardoso-neto/archive-chan/blob/1e0c3983b807d0b151e879605e233e5710ec5e5a/src/archive_chan/extractors/fourchan_api.py#L245}}.
Assim, ao final do download de uma thread, um página \texttt{.html} é renderizada com o conteúdo da thread e salva.

Por causa da possibilidade do uso de front-ends diferentes, tornamos a renderização opcional e pretendemos desacoplá-la do archive-chan no futuro como visto no issue \#6\footnote{\url{https://github.com/cardoso-neto/archive-chan/issues/6}}.

\subsection{Arquitetura genérica a chans}

Chans são um tipo de fórum online baseado em imagens (um imageboard).
Alguns exemplos são o \url{27chan.org}, o \url{dogolachhhnaqa7n.onion}, o \url{99chan.org}, e o \url{endchan.net}. 
As ferramentas que desenvolvemos foram desenvolvidas com suporte a fácil extensibilidade.
Por meio de herança e polimorfismo no archive-chan (pode ser visto na classe pai \texttt{Extractor} e filho \texttt{FourChanExtractor}) e nos outros sendo completamente genéricos dependendo apenas do \texttt{chan-url-lib} (que também foi desenvolvido na expectativa de agnóstica a chans) suportar o encoding das URLs dos chans que se queira suportar.

\chapter{Testes e Validações}

Neste capítulo serão descritas as etapas de testes e validações das ferramentas implementadas.

\section{Testes Unitários}

Essa seção descreve os teste unitários desenvolvidos para cada ferramenta.

\subsection{Archive-Chan}
Foram realizados testes unitários para os métodos da ferramenta archive-chan, e juntados num projeto de testes que acompanha a solução no mesmo repositório no Github \url{https://github.com/cardoso-neto/archive-chan}.
Os testes foram feitos na linguagem Python com auxilio da biblioteca de testes PyUnit.
Eles foram feitos com o objetivo de testar o funcionamento individual de partes da ferramenta e possibilitam a identificação rápida de bugs e defeitos, analisando as entradas e saídas desses métodos.
É vericada a criação do arquivos do 4chan na máquina local e o sucesso das chamadas à API do 4chan.

\subsection{Network-Chan}

Foram realizado testes unitários para cada método da ferramentas Network-Chan, e juntados num projeto de teste que acompanha a solução no mesmo repositório no GitHub \url{https://github.com/cardoso-neto/network-chan}.
Os testes foram feitos na linguagem Python com auxilio da biblioteca de testes PyUnit.
Eles foram feitos com o objetivo de testar o funcionamento individual de partes da ferramenta e possibilitam a identificação rápida de bugs e defeitos, analisando as entradas e saídas desses métodos.

Devido ao projeto utilizar uma biblioteca para se comunicar com o IPFS, não há a possibilidade de testar diretamente a conexão com o IPFS por isso a importância dos testes de integração (das saídas e entradas) da solução. 

A ferramenta Network-chan tem duas funções principais, adicionar e baixar um arquivo ao IPFS, os testes atuam diretamente nessas etapas verificando a adesão dos arquivos ao IPFS no caso de adicionar arquivo e conferindo a criação do arquivo na máquina no caso de baixar o arquivo.

\section{Validação com usuário}\label{sec:val-usr}

Essa seção descreve as validações feitas junto com os usuários para a ferramenta \texttt{archive-chan} e apenas essa, pois se provou difícil encontrarmos candidatos dispostos aos testes.

\subsection{Archive-Chan}

No começo de junho, começamos a busca por possíveis alpha testers das nossas ferramentas.
Nossos contatos todos foram conduzidos em canais públicos (Github e Gitter).
Focamos nossos esforços em usuários que estivessem comentando ou desenvolvendo scrapers para o 4chan, pois seriam o público mais interessado no nosso ponto de vista.
Conseguimos feedback de três dos usuários que contatamos.
Importante ressaltar o pouco tempo de procura e quão nicho é o nosso público-alvo atualmente (sabe rodar programas no terminal, gosta de chans, gosta de arquivar páginas da internet offline).

Enviamos aos usuários o link do repositório da ferramenta com o intuito de que dependessem apenas da documentação já disponível no readme e no \texttt{--help} e todos a acharam satisfatória, no entanto todos estavam bastante acostumados com aplicativos de linha de comando.

As subsubseções abaixo serão identificadas pelo nickname do Github de cada um dos usuários, pois todas nossas conversas foram públicas e isso facilitará a verificação das informações que trouxemos.

\subsubsection{LameLemon}

O usuário do Github \texttt{@LameLemon} se comunicou bem pouco e primariamente através de issues no Github\footnote{\url{https://github.com/LameLemon/archive-chan/issues?q=is\%3Aissue}}.
Ele começara um projeto de scraper para o 4chan em Março de 2019, mas o abandonou em seguida.
Ele incorporou ao seu projeto duas features do archive-chan que nós desenvolvemos:

\begin{itemize}
    \item Download de threads arquivadas mediante a argumento opcional na CLI\footnote{\url{https://github.com/LameLemon/archive-chan/commit/f69beab6555fa390955feeb45442a1f564136725}}
    \item Refatoração da lógica de extração das URLs das threads de uma board para fora do \texttt{main}\footnote{\url{https://github.com/LameLemon/archive-chan/commit/efd3f22abfa22fc944dfd944c01a485ecfb6884f}}
\end{itemize}

\subsubsection{dermalikmann}

O usuário do Github \texttt{@dermalikmann} sentiu falta de duas features:

\begin{itemize}
    \item download de threads em batches pela CLI;
    \item monitoramento com um timer de um conjunto de threads definido pelo usuário.
\end{itemize} 

\subsubsection{baraa272}

O usuário do Github \texttt{@baraa272} sentiu falta das seguintes features:

\begin{itemize}
    \item Suporte ao Windows
        \subitem Por causa de alguns problemas com o GPG (programa que usamos para verificar se as hashes das fotos baixadas batem), o usuário teve que criar uma máquina virtual com GNU/Linux.
    \item Um auxílio para indexar as threads
        \subitem Como o archive-chan salva as threads em pastas com apenas a ID da thread (sem seu título), fica difícil de um humano navegar sem um auxílio.
        \subitem Implementamos então o \texttt{thread-indexer}\footnote{\url{https://github.com/cardoso-neto/archive-chan/tree/master/src/thread_indexer}} que cria um JSON mapeando as IDs das threads para seus títulos sanitizados (e.g.: \texttt{\{224624183: "super-cub"\}}.
    \item Atualização automática de boards
        \subitem Às vezes novas boards são criadas, então idealmente o archive-chan seria agnóstico a boards, mas não é o caso atualmente.
        \subitem O usuário queria usar em boards que nós não tínhamos adicionado ainda, então ele mesmo editou o código fonte e mandou um patch com as novas boards para nós que pode ser visto aqui: \url{https://github.com/cardoso-neto/archive-chan/commit/50eec0d5d939fb90f59fdbc6886f1b2914fb694c}
    \item Suporte a versões de python mais recentes
        \subitem o usuário queria poder usar a versão que veio na distro dele (python 3.8), mas o archive-chan tem dependências que só funcionam em python 3.7.
\end{itemize}

Nossa interação com o usuário pode ser lida em sua inteiridade na sala pública do Gitter em que ela aconteceu: \url{https://gitter.im/archive-chan/community}

\chapter{Conclusão}

Como conclusão deste trabalho, identificamos que atingimos o objetivo de utilizar os dados do 4chan de forma local e compartilhá-los de forma descentralizada.
As ferramentas se mostraram satisfatórias em termos de performance e usabilidade, mas reconhecemos a necessidade de uma simplificação considerável para algo como uma extensão do navegador caso queiramos adoção das massas.
Para o 4chan, haverá ganhos de redundância e registros históricos.
Para os usuários, embora ainda não seja uma aplicação totalmente descentralizada, trará os principais benefícios que a descentralização traz: resistência à censura, conteúdo permanente, e maior largura de banda para downloads grandes (pois pode ser paralelizado de vários peers diferentes).

Portanto, vemos que a solução demonstra um modo diferente de utilizarmos a web; mitigando problemas do sistema anterior, mas sem sacrificar o conteúdo e a base de usuários como mídias sociais distribuídas costumam fazer.
Assim, podemos afirmar que a solução cumpre os objetivos de encontrar os elementos que devem ser considerados para propor um sistema que transforme a arquitetura de software centralizada das mídias sociais da web 2.0 em uma arquitetura descentralizada baseada na web 3.0.

\section{Limitações}

Como limitações destacamos que a solução proposta neste trabalho ainda é dependente do 4chan para funcionar, visto que ela consome os dados do 4chan e sem ele apenas os dados já baixados poderão ser compartilhados na rede peer-to-peer.
Isso do ponto de vista desse trabalho não é um problema; isso é \textit{by design}, mas merece atenção por um ponto central de falha que pode ter seu domínio suspenso por exemplo.

Identificamos também que o armazenamento poderá ser um fator limitante da solução visto que os usuários são responsáveis pelo armazenamento dos dados se não quiserem depender do servidor central.
Assim os dados dependem dos usuários para existir. Mesmo com os usuários podendo dividir a load do armazenamento entre si (i.e., a rede não força ninguém a baixar todos os dados, ou alguma parcela específica) e os dados estarem espalhados por diversos nós, o armazenamento ainda terá a limitação de quanto os usuários estão dispostos a dedicar do seu armazenamento e do quanto estão dispostos a servir esses dados.
Nós não desenvolvemos nenhum mecanismo de rastreamento que possa dizer quantas cópias de um arquivo existem na rede com precisão para que alguém saiba se pode apagar um arquivo para liberar espaço com segurança ou não.

Existem também aspectos relacionados à usabilidade que exigem um refinamento na solução para facilitar o uso por usuários menos técnicos (como a eliminação de algumas dependências externas de difícil instalação para leigos comentadas na seção \ref{sec:val-usr}).
Sobre desempenho, não pudemos executar uma bateria de testes em larga escala para averiguarmos a escalabilidade da DHT a medida que os pares chave-valor aumentam e a medida que nós entrando e saindo da rede se tornam mais frequentes.


\section{Trabalho Futuros}

Listamos, nessa seção, melhorias que vemos como úteis para o melhor proveito dessa nova arquitetura que propomos.

\paragraph{Sistema de governança}

A definição dos formatos e estruturas de dados precisa ser gerida de forma aberta e transparente.
O suporte a novas features em mídias sociais frequentemente precisará de novos campos nas estruturas padronizadas, então elas precisam ser facilmente extensíveis para evitar forks excessivos na rede.
Imaginamos algo envolvendo discussões no Github como o IPFS faz ou discussões em um fórum próprio como Bitcoin e o git-annex fazem.

\paragraph{Ecossistema web 3.0}

A medida que outros scrapers forem sendo desenvolvidos (para o youtube, instragram, tiktok), acreditamos que eles trarão mais valor se integrados uns aos outros.
Por exemplo, um post no 4chan pode conter links para um vídeo do Youtube.
Se esse vídeo não for baixado também, ele pode ser censurado ou removido.
Então precisamos de um módulo extensível que faça o parsing de cada post à procura de URLs para as quais o usuário tenha instalado um scraper, a fim de preservar a totalidade das dicussões.

\paragraph{Lightclient}

Um cliente leve (que não seja server, mas apenas cliente) para a rede peer-to-peer, a fim de não onerar a banda de upload de usuários que não queiram disponibilizar seus recursos para outros.

