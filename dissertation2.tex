
\chapter{Arquitetura aplicada: 4chan.org}

Com o objetivo de instanciar um exemplo concreto da estratégia de transição que propomos, implementaremos as ferramentas acima propostas de forma que permitam o armazenamento e compartilhamento de conteúdo do 4chan de forma descentralizada.
Brevemente, essas ferramentas serão:
\begin{enumerate*}[label=(\arabic*)]
    \item Um \textit{scraper} confiável e de de alto desempenho,
    \item um módulo de \textit{networking} capaz de localizar quais peers possuem quais dados, e
    \item um módulo de criptografia para assinar e verificar assinaturas, a fim de montarmos a \textit{internet of trust}. 
\end{enumerate*}

\section{Requisitos da Solução}

Neste capítulo serão apresentados os Requisitos Funcionais e Requisitos Não Funcionais da solução para separar a camada de dados da camada de apresentação do 4chan.

Segundo Sommerville (2011, p. 57)~\cite{SOMMERVILLE1}, os requisitos de um sistema são:

\begin{directcite}
as descrições do que o sistema deve fazer, os serviços que
oferece e as restrições a seu funcionamento.
Esses requisitos refletem as necessidades dos clientes para
um sistema que serve a uma finalidade determinada, como controlar
um dispositivo, colocar um pedido ou encontrar informações.
\end{directcite}

\subsection{Requisitos Funcionais}

Os requisitos funcionais de um sistema representam as necessidades que o sistema deverá suprir e todas as funções que o sistema deve ter.
Sobre os requisitos funcionais de uma solução, Sommerville (2011, p. 59) considera que~\cite{SOMMERVILLE1}:

\begin{directcite}
São declarações de serviços que o sistema deve fornecer,
de como o sistema deve reagir a entradas específicas e
de como o sistema deve se comportar em determinadas situações.
Em alguns casos, os requisitos funcionais também
podem explicitar o que o sistema não deve fazer.
\end{directcite}

De acordo com o nosso entendimento, a solução possui os seguintes requisitos funcionais.

\begin{enumerate}
    \item Módulo de \textit{scraping}:
    \begin{enumerate}[label*=\arabic*.]
        \item baixar todo o texto de uma thread.
            \subitem escolher uma thread passando sua URL por argumento.
            \subitem saber quando o download foi concluído.
        \item baixar toda a mídia (fotos, vídeos) de uma thread.
            \subitem escolher uma thread passando sua URL por argumento.
            \subitem monitorar o progresso do download.
            \subitem poder interromper um download.
            \subitem continuar um download de onde parou.
        \item baixar todas threads de um board.
            \subitem escolher um board passando seu nome como argumento.
            \subitem monitorar quais threads estão sendo baixadas.
            \subitem baixar threads utilizando computação paralela.
            \subitem interromper download.
            \subitem retomar download.
        \item reconhecer quando uma thread estiver fechada e marcá-la como tal para evitar desperdício de recursos.
            \subitem reconhecer threads arquivadas.
            \subitem reconhecer threads apagadas.
            \subitem marcar como tal no banco de dados.
        \item monitorar threads por novos posts e baixá-los assim que postados.
            \subitem opcionalmente receber um argumento que defina de quanto em quanto tempo checar por novos posts numa thread.
        \item monitorar boards por novos posts e adicioná-las a fila de monitoramento de threads assim que criadas.
            \subitem opcionalmente receber um argumento que defina de quanto em quanto tempo checar por novas threads num board.
        \item converter os dados baixados em um formato padronizado.
            \subitem opcionalmente receber um argumento para exportar threads.
            \subitem permitir que o usuário especifique qual dos padrões implementados ele deseja.
            \subitem utilizar computação paralela na exportação.
    \end{enumerate}
    \item Módulo de \textit{networking}:
    \begin{enumerate}[label*=\arabic*.]
        \item Requisitar conteúdo para outros peers.
        \item Servir conteúdo local para outros peers.
    \end{enumerate}
    \item Módulo de criptografia:
    \begin{enumerate}[label*=\arabic*.]
        \item Criar um novo par de chaves (privada e pública) criptográficas.
        \item Escolher em quais peers confio para baixar conteúdo que não esteja disponível no servidor do 4chan.
        \item Exportar minha chave pública em cleartext para compartilhamento.
        \item Assinar arquivos com minha chave privada.
        \item Verificar se assinatura de arquivos correspondem a alguma chave pública que tenho marcada como confiável.
    \end{enumerate}
\end{enumerate}

\subsection{Requisitos Não Funcionais}

Sobre os requisitos não funcionais de uma solução, Sommerville (2011, p. 59) considera que~\cite{SOMMERVILLE1}:

\begin{directcite}
Os requisitos não funcionais, como o nome sugere, são requisitos
que não estão diretamente relacionados com os serviços específicos
oferecidos pelo sistema a seus usuários.
Eles podem estar
relacionados às propriedades emergentes do sistema, como
confiabilidade, tempo de resposta e ocupação de área.
\end{directcite}

Em outras palavras, os requisitos não funcionais de um sistema representam as necessidades internas do sistema, necessidades estas que devem ser de compreensão do desenvolvedor, resumindo-se aos itens de segurança, usabilidade, confiabilidade, desempenho, hardware e software.
De acordo com a análise realizada, o sistema possui os seguintes requisitos não funcionais:

\begin{enumerate}
    \item \textbf{Desempenho} Ter desempenho em termos de \textit{response time} similares às redes que já estou acostumado.
    \item \textbf{Tecnologia} Empregar um formato de dados base ado em padrões livres e \textit{open-source}.
    \item \textbf{Desempenho} Utilizar estruturas de dados que permitam downloads de granularidade pequena (não quero ter que baixar um \textit{board} inteiro se eu só quiser uma \textit{thread}).
    \item \textbf{Segurança} Utilizar estruturas de dados que possam ser replicadas e atualizadas independentemente e paralelamente num rede peer-to-peer sem a necessidade de coordenação entre as réplicas nem uma entidade central (CRDTs SIGLA + CITAR \url{https://en.wikipedia.org/wiki/Conflict-free_replicated_data_type}
    \item \textbf{Desempenho} O formato de dados deve evitar duplicação de conteúdo, a fim de economizar recursos de armazenamento.
    \item \textbf{Ambiente} Ser compatíveis com distribuições GNU/Linux.
    \item \textbf{Usabilidade} Respeitar a filosofia UNIX.
\end{enumerate}
\begin{figure}[H]
    \centering
    \includesvg[height=0.65\textheight]{fig/RequisitosNaoFuncionais.svg}
    \caption[Requisitos Não-Funcionais]{\label{fig:Requisitos_Nao_Funcionais}
        Requisitos Não-Funcionais\\
        Fonte: os autores.
    }
\end{figure}
\section{Tecnologias que utilizaremos}

Nesta seção, falaremos sobre as tecnologias que utlizamos 

\subsection{Python multiprocessing}
https://docs.python.org/3/library/multiprocessing.html

\subsection{Python requests}
https://docs.python-requests.org/en/master/

\subsection{Python setuptools}
https://setuptools.readthedocs.io/en/latest/

\subsection{Python toolz}
https://github.com/pytoolz/toolz
Functional programming library.

\subsection{\label{sec:ipfs}InterPlanetary File System}

IPFS is a peer-to-peer distributed file system and hypermedia protocol~\cite{IPFS}.
It provides a high throughput content-addressed block storage model, with content-addressed hyper links; all through its core data format called IPLD (InterPlanetary Linked Data) which is going to be further explained in subsection~\ref{subsec:ipld}.
Juan Benet, IPFS's author, often describes it in his talks as a single BitTorrent swarm\ref{bittorent-swarm} with peers exchanging objects within a Git repository~\cite{github:ipfspaper}.
This is simple, informal, and incomplete, but it is a powerful analogy that captures the essential mechanics of IPFS.
\longfootnote[bittorent-swarm]{
    Together, all peers (including seeds) sharing a torrent are called a swarm.
}
IPFS has no single point of failure, nodes do not need to trust each other, and all peers are independent.
The following paragraphs will go into each relevant subsystem of IPFS.

\subsubsection{\label{subsec:ipld}IPLD}

IPLD is a common hash-chain format for distributed data structures that are universally addressable and linkable~\cite{github:ipld}.
All IPFS data is stored and transferred in this format.
Shortly said, it is a Merkle DAG where links between objects are cryptographic hashes of the objects' to which they point.
An IPLD implementation offers developers the ability to traverse it using Unix-style paths when the Merkle DAG has named Merkle links, a data model to describe Merkle DAGs that is flexible, JSON-inspired, and self-describing, and standard serialization algorithms for different formats (e.g., JSON, CBOR, Protobuf, RDF).
These structures allow us to do for data what URLs and links did for HTML web pages.
It is also worth emphasizing that no feature is lost when using this data format.

\subsubsection{\label{subsec:libp2p}libp2p}

An ever-growing, fully modular, flexible, and extensible network stack; a collection of peer-to-peer protocols geared towards multi-platform interoperability with high-latency scenarios considered and easy upgrades in mind.
Protocols for finding peers and connecting to them, and for finding content and transferring it.
All while keeping it secure by not relying on centralized registers and third parties (allowing users to work offline or work on their LAN only), as well as enabling encrypted connections by default.

One of the main goals of developing libp2p separately from IPFS is so that developers no longer need to reinvent the P2P networking wheel yet again in the near-future.
Each big Web3.0 project like BitTorrent or DAT had to write their networking code mostly from scratch.
Juan Benet often refers to IPFS as a thin waste protocol\ref{foot:thinWaist}; meaning the data structures are what needs to be standardized and everything else will communicate.
\longfootnote[foot:thinWaist]{
    The term “thin-waist” has been used because there is a vast diversity of applications and hardware that sit above and below the thin waist of universally shared control mechanisms (IPLD). Further details on network architecture and protocols can be found on this Caltech wiki page by Doyle et al~\cite{INTERNETWAIST} and, more formally, on a 2005 paper by Doyle as well on the robustness of the internet~\cite{INTERNETROBUSTNESS}.
}

\subsection{GNU Privacy Guard}
gnupg.org

\section{Implementação}

Focaremos nossos esforços de implementação inteiramente no scraper, pois é o mais relevante para fins de demonstração do conceito e de contribuição open-source.
Graças aos excelentes esforços da Protocol Labs no IPFS~\ref{label da seção que explica o ipfs}, poderemos nos apoiar nos ombros de gigantes no tocante a questão de \textit{networking} realizando apenas de integrações simples com a API da libp2p~\ref{subsec:libp2p} e da IPLD~\ref{subsec:ipld}.

\subsection{Modelo de dados}

\begin{tikzpicture}[auto,node distance=1.5cm]
    \node[entity] (board_node) {Boards}
        [grow=up,sibling distance=3cm]
        child {node[attribute] {Name}}
        child {node[attribute] {Abbreviation}};
    \node[relationship] (board_thread) [below right = of board_node] {have};
    \node[entity] (thread_node) [above right = of board_thread]	{Threads}
        child {node[attribute] {id}};
    \path (board_thread) edge node {1} (board_node) edge node {N} (thread_node);
    \node[entity] (post_node) [above = of thread_node] {Posts}
        [grow=up,sibling distance=3cm]
        child[grow=left,level distance=3cm] {node[attribute] {id}}
        child {node[attribute] {Author name}}
        child {node[attribute] {Media}}
        child {node[attribute] {Text}};
    \node[relationship] (thread_post) [above right = of thread_node] {have};
    \path (thread_post) edge node {1} (thread_node) edge node {N} (post_node);
\end{tikzpicture}

\subsection{Banco de dados}

Para o nosso banco de dados, escolhemos o que há de mais portátil e acessível: o sistema de arquivos.
Os dados colaboram para isso, pois não têm nenhuma relação com cardinalidade que nos impeça de modelá-los denormalizados, então assim o fizemos. Como pode ser visto na árvore de diretórios na \textbf{figura X}, temos:

\begin{itemize}
    \item Uma pasta raíz com N subpastas, cada um correspondendo a um board.
    \item Cada pasta de board, identificada pela abreviação única do board, possui N subpastas e cada uma dessas corresponde a uma thread.
    \item Cada pasta de thread possui 
        \subitem uma subpasta \code{media} onde todas imagens e vídeos da thread são armazenados;
        \subitem um arquivo \code{thread.json} onde todos dados pertinentes àquela thread são armazenados.
\end{itemize}

% \begin{verbatim}
% $ tree 4chan-archive/
% └── w/
%     ├── 2131136/
%     │   ├── media/
%     │   │   ├── 1600798279711.png
%     │   │   ├── 1600814639570.jpg
%     │   │   └── ...
%     │   └── thread.json
%     ├── 2180136/
%     │   ├── media/
%     │   └── thread.json
%     └── 2180395/
%         ├── media/
%         └── thread.json
% \end{verbatim}

\chapter{Modelagem do Sistema}

Este capítulo apresenta a modelagem completa das ferramentas
propostas, buscando o esclarecimento de suas funcionalidades e características
através de sua abstração.

\section{Diagramas de Atividades}

Nesta seção serão apresentados os diagramas de atividade das ferramentas quem tem contato direto com o usuário(\textit{Scraper} e Rede \textit{Peer-to-Peer}), com foco nas atividades que serão executadas no funcionamento destas ferramentas.
\subsection{Extração de Dados\label{subsec:extracao_de_dados}}
Esse processo realiza a extração de dados da web 2.0, esse é o método principal para a migração de dados da web 2.0 para a web 3.0, as principais atividade do processo podem ser vistas no diagrama de atividades representado na figura ~\ref{fig:extracao_de_dados}.
\begin{figure}[H]
    \centering
    \includesvg[width=\textwidth]{fig/Extracao_de_dados.svg}
    \caption[Diagrama de Atividades : Extração de Dados]{\label{fig:extracao_de_dados}
        Diagrama de Atividades : Extração de Dados\\
        Fonte: os autores.
    }
\end{figure}
Esse processo já está descrito a partir da rede social da web 2.0 que será utilizada para a aplicação das ferramentas deste trabalho, assim além
do usuário e da ferramenta \textit{scraper} temos a API do \textit{4chan}. Por estarmos utilizando o \textit{4chan} que é uma plataforma que não necessita o cadastro de um usuário não existe o passo para a autenticação junto a API, para outras plataformas como \textit{Facebook} e \textit{Twitter} necessitaríamos um passo a mais e a utilização  de um \textit{token} nas chamadas a API.

O processo começa com a etapa onde o usuário indica qual o dado deseja salvar em seu computador; Exemplo: \textit{board}, \textit{thread}, comentário (Todos exemplos de postagens do \textit{4chan}). E depois o
formato que deseja que eles sejam salvos; Exemplo: XML, JSON, TXT. Após essa primeira etapa o \textit{scraper} recebe esses dados e cria uma requisição para a API do \textit{4chan} e fica no aguardo da
reposta, recebido a resposta o \textit{scraper} formata os dados conforme os parametros que o usuário especificou e salva os dados no local indicado pelo usuário.

\subsection{Solicitação de Dados pela Rede P2P\label{subsec:Solicitacao_Dados}}
Este processo realiza a comunicação e a troca de dados dentro da rede P2P, uma fez extraído os dados da web 2.0 esse vai ser o novo formato de obtenção e troca de dados sem a necessidade do 
do \textit{4chan}. A partir deste processo o usuário fica independente das plataformas da web 2.0. as principais atividade do processo podem ser vistas no diagrama de atividades 
representado na figura ~\ref{fig:Solicitacao_Dados}. 
\begin{figure}[H]
    \includesvg[width= 1.11\textwidth]{fig/Solicitacao_Dados.svg}
    \caption[Diagrama de Atividades : Solicitação de Dados pela Rede P2P]{\label{fig:Solicitacao_Dados}
        Diagrama de Atividades : Solicitação de Dados pela Rede P2P\\
        Fonte: os autores.
    }
\end{figure}
O processo começa com a etapa onde o usuário indica qual o dado deseja acessar; Exemplo:\textit{board}, \textit{thread}, comentário (Todos exemplos de postagens do \textit{4chan}). Depois a rede procura pela \textit{hash} desse \textit{post} na rede para descobrir qual nodo tem esse dado, verificando-se existem nodos da lista de nodos confiáveis do usuário, não tendo nodos da lista confiáveis oferece a opção de baixar
de um nodo não confiável. Após essa primeira etapa é realizado a comunicação entre os nodos e transferido os arquivos, por fim o usuário escolhe se deseja manter salvo os arquivos 
em usa máquina. 
\section{Diagramas de Sequência}
Nesta seção serão apresentados os diagramas de sequencia das ferramentas quem tem contato direto com o usuário(\textit{Scraper} e Rede \textit{Peer-to-Peer}), com foco na comunicação entre os objetos e nos tempos de execução e reposta das funcionalidade destas ferramentas.
\subsection{Extração de Dados}
Esta seção faz a analise dos tempos de execução e resposta do processo de extração de dados descrito na seção~\ref{subsec:extracao_de_dados}. A representação dos tempos de execução podem ser vistas através do diagrama de sequencia representado na figura~\ref{fig:seq_extracao_dados}. 
\begin{figure}[H]
    \includesvg[width= 1.10\textwidth]{fig/DiagramadeSequenciaScraper.svg}
    \caption[Diagrama de Sequência : Extração de Dados]{\label{fig:seq_extracao_dados}
        Diagrama de Sequência : Extração de Dados\\
        Fonte: os autores.
    }
\end{figure}

O processo inicia com a solicitação síncrona do usuário ao \textit{Scraper}, que envia uma requisição síncrona para a API do \textit{4chan}, o \textit{Scraper} fica a espera de uma resposta assíncrona do \textit{4chan} e envia uma resposta assíncrona para o usuário. 
O tempo de espera do usuário deve ser perto do tempo de espera de uma aplicação da web 2.0, visto que o processo é o mesmo. A única diferença de tempo que pode ser esperada depende do formato do dado que o usuário escolheu
se o formato for diferente da resposta da API um tempo de formatação do \textit{Scraper} deverá ser adicionado ao tempo de espera esperado.

\subsection{Solicitação de Dados pela Rede P2P}
Esta seção faz a analise dos tempos de execução e resposta do processo de solicitação de dados pela \textit{rede peer-to-peer} descrito na seção~\ref{subsec:Solicitacao_Dados}. A representação dos tempos de execução podem ser vistas através do diagrama de sequencia representado na figura~\ref{fig:seq_Solicitacao_dados}. 
\begin{figure}[H]
    \includesvg[width= 1.11\textwidth]{fig/DiagramaSequenciaRedeP2P.svg}
    \caption[Diagrama de Sequência :Solicitação de Dados pela Rede P2P]{\label{fig:seq_Solicitacao_dados}
        Diagrama de Sequência : Solicitação de Dados pela Rede P2P\\
        Fonte: os autores.
    }
\end{figure}
O processo inicia com solicitação síncrona do usuário a rede \textit{peer-to-peer}, indicando os dados
requisitados, a rede \textit{peer-to-peer} envia uma resposta assíncrona que pode ser dado encontrado ou dado não encontrado,
a partir desse dado o usuário envia uma solicitação ao \textit{peer} que contém o conteúdo desejado
e começa o download do conteúdo. Este processo contém algumas variáveis, como a conexão entre os dois \textit{peers}
e quantos \textit{peers} na rede contém aquele dado, pois na rede \textit{peer-to-peer} é possível baixar partes de um dado de fontes diferentes. Assim o tempo de execução pode ser menor ou maior que o da plataforma da web 2.0.
\section{Diagramas de Casos de Uso}
Nesta seção serão apresentados os diagramas de caso de uso e os atores
modelados por ferramenta. O objetivo do uso desta modelagem é abordar mais profundamente as funcionalidades das ferramentas atráves da modelagem de casos de uso. 
\subsection{Atores}
Por o projeto ser um conjunto de ferramentas, elas serão utilizadas apenas por um tipo de Ator, o usuário. Ele será responsável por fornecer os dados de entrada para as ferramentas, e as configurações que melhor desejar e que possam ser fornecidas as ferramentas.
\begin{figure}[H]
    \centering
    \includesvg[width=0.09\textwidth]{fig/Ator.svg}
    \caption[Diagrama do Ator]{\label{fig:diagrama_do_ator}
        Diagrama do Ator\\
        Fonte: os autores.
    }
\end{figure}
\subsection{Casos de Uso Scraper}
Nesta seção será apresentado os casos de uso para o ator usuário na ferramenta \textit{Scraper}.
\subsubsection{UC01 - Especificar o Formato dos Dados}
Permite ao usuário especificar ao \textit{Scraper} qual o formato dos dados que ele espera receber. Para isso o usuário deve passar como parâmetro para o \textit{Scraper} essa informação, ou configurá-lo previamente com o formato favorito.
\subsubsection{UC02 - Baixar conteúdo de Texto de uma Thread do 4chan}
Permite ao usuário baixar todo o texto de uma thread do 4chan. Para isso o usuário deve passar como parâmetro para o \textit{Scraper} a id da thread que deseja baixar o texto.
\subsubsection{UC03 - Baixar conteúdo de mídia(fotos, vídeos)  de uma Thread do 4chan}
Permite ao usuário baixar todo a mídia de uma thread do 4chan. Para isso o usuário deve passar como parâmetro para o \textit{Scraper} a id da thread que deseja baixar a mídia.
\subsubsection{UC04 - Baixar todo o conteúdo de uma Thread do 4chan}
Permite ao usuário baixar todo a conteúdo de uma thread do 4chan (mídia e texto). Para isso o usuário deve passar como parâmetro para o \textit{Scraper} a id da thread que deseja baixar todo o conteúdo.
\begin{figure}[H]
    \centering
    \includesvg[width=0.6\textwidth]{fig/Ator_Scraper1.svg}
    \caption[Casos de uso Scraper UC01,UC02,UC03,UC04]{\label{fig:Ator_Scraper1}
        Casos de uso Scraper UC01,UC02,UC03,UC04\\
        Fonte: os autores.
    }
\end{figure}
\subsubsection{UC05 - Especificar Threads para monitoramento}
Permite ao usuário informar uma thread para monitoramento no 4chan, assim o scraper fará consultas constantemente, buscando e baixando alterações na thread. Para isso o usuário deve passar como parâmetro para o \textit{Scraper} a id da thread que deseja monitorar.  
\subsubsection{UC06 - Especificar Boards para monitoramento}
Permite ao usuário informar um board para monitoramento no 4chan, assim o scraper fará consultas constantemente, buscando e baixando alterações no board. Para isso o usuário deve passar como parâmetro para o \textit{Scraper} a id do board que deseja monitorar.  
\subsubsection{UC07 - Interromper o monitoramento de uma Thread que estiver fechada}
Permite ao usuário configurar o \textit{Scraper} para parar de monitorar threads que estiverem fechadas.
\begin{figure}[H]
    \centering
    \includesvg[width=0.65\textwidth]{fig/Ator_Scraper2.svg}
    \caption[Casos de uso Scraper UC05,UC06,UC07]{\label{fig:Ator_Scraper2}
        Casos de uso Scraper UC05,UC06,UC07\\
        Fonte: os autores.
    }
\end{figure}
\subsection{Casos de Uso Networking}
Nesta seção será apresentado os casos de uso para o ator usuário com ferramenta \textit{Networking}.
\subsubsection{UC01 - Procurar Conteúdo pela Rede}
Permite ao usuário procurar conteúdo pela rede peer-to-peer, para isso ele deve passar o nome ou hash do arquivo que deseja.
\subsubsection{UC02 - Requisitar Conteúdo pela Rede}
Permite ao usuário requisitar conteúdo pela rede peer-to-peer, para isso ele deve passar o nome ou hash do arquivo que deseja.
\subsubsection{UC03 - Fornecer Conteúdo pela Rede}
Permite ao usuário fornecer conteúdo pela rede peer-to-peer, para isso ele deve ter o arquivo e configurar a rede para fornecer o conteúdo.
\begin{figure}[H]
    \centering
    \includesvg[width=0.5\textwidth]{fig/Ator_Networking1.svg}
    \caption[Casos de uso Networking UC01,UC02,UC03]{\label{fig:Ator_Networking2}
        Casos de uso Networking UC01,UC02,UC03\\
        Fonte: os autores.
    }
\end{figure}
\subsection{Casos de Uso Criptografia}
Nesta seção será apresentado os casos de uso para o ator usuário com ferramenta \textit{Criptografia}.
\subsubsection{UC01 - Criar um par de chaves criptográficas}
Permite ao usuário criar um par de chaves criptográficas para garantir a autenticidade e integridade no compartilhamento de arquivos na rede.
\subsubsection{UC02 - Criar lista de nodos confiáveis}
Permite ao usuário criar uma lista de nodos confiáveis, para isso o usuário deve armazenar uma lista com a chave pública dos nodos que julgar confiáveis.
\subsubsection{UC03 - Exportar minha chave pública para Compartilhamento}
Permite ao usuário exportar a chave pública no formato cleartext para compartilhamento. Para isso o usuário necessita ter um par de chaves criptográficas. 
\begin{figure}[H]
    \centering
    \includesvg[width=0.7\textwidth]{fig/Ator_Crypto1.svg}
    \caption[Casos de uso Criptografia UC01,UC02,UC03]{\label{fig:Ator_Crypto1}
        Casos de uso Criptografia UC01,UC02,UC03\\
        Fonte: os autores.
    }
\end{figure}
\subsubsection{UC04 - Assinar arquivos com a minha chave privada}
Permite ao usuário assinar arquivos com sua chave privada para que os outros peers da rede que o mantém como nodo confiável e tem sua chave publica possam reconhecer a assinatura e identificar que ele é um peer confiável. Para isso o usuário deve ter um par de chaves criptográficas.
\subsubsection{UC05 - Verificar se a assinatura de um arquivo pertence a minha lista de nodos confiáveis}
Permite ao usuário identificar a assinatura de um nodo confiável a partir da chave publica na lista de nodos confiáveis. Para isso o usuário necessita ter armazenado na sua lista de nodos confiáveis a chave publica do nodo.
\begin{figure}[H]
    \centering
    \includesvg[width=0.5\textwidth]{fig/Ator_Crypto2.svg}
    \caption[Casos de uso Criptografia UC04,UC05]{\label{fig:Ator_Crypto2}
        Casos de uso Criptografia UC04,UC05\\
        Fonte: os autores.
    }
\end{figure}
\section{Detalhamento dos Casos de Uso}

\subsection{Especificar o Formato dos Dados}
\subsubsection{Descrição}
Permite ao usuário especificar ao \textit{Scraper} qual o formato dos dados que ele espera receber. Para isso o usuário deve passar como parâmetro para o \textit{Scraper} essa informação, ou configurá-lo previamente com o formato favorito.
\subsubsection{Atores : Usuário}
\subsubsection{Pré-Condição}
O usuário deve informar o formato dos dados.
\subsubsection{Pós-Condição}
O usuário tem o dados no formato desejado.
\subsubsection{Fluxo Principal}
\begin{enumerate}
    \item Abre o Scraper pela linha de comando;
    \item Passa como parâmetro o id do dado que deseja e o formato dos dados;
    \item Scraper realiza a busca do dado;
    \item Scraper realiza a formatação do dado;
    \item O usuário abre o dado;
\end{enumerate}
\subsubsection{Fluxo Alternativo}
\begin{itemize}
    \item Dado não existe no 4chan: ocorre quando no passo 3 scraper não acha o dado indicado pelo usuário no 4chan.
    \begin{enumerate}
        \item Mostra mensagem de erro;
        \item Retorna ao passo 2.
    \end{enumerate}
    \item Scraper não reconhece o formato especificado: ocorre quando no passo 4 o scraper vai formatar o dado indicado pelo usuário e não consegue identificar o formato desejado.
    \begin{enumerate}
        \item Mostra mensagem de erro;
        \item Retorna ao passo 2.
    \end{enumerate}
\end{itemize}


\subsection{Baixar conteúdo de Texto de uma Thread do 4chan}
\subsubsection{Descrição}
Permite ao usuário baixar todo o texto de uma thread do 4chan. Para isso o usuário deve passar como parâmetro para o \textit{Scraper} a id da thread que deseja baixar o texto.
\subsubsection{Atores : Usuário}
\subsubsection{Pré-Condição}
O usuário deve informar o id das threads que deseja salvar.
\subsubsection{Pós-Condição}
O usuário terá o todo conteúdo de texto da thread indicada.
\subsubsection{Fluxo Principal}
\begin{enumerate}
    \item Abre o Scraper pela linha de comando;
    \item Passa como parâmetro o id do dado que deseja e o formato dos dados;
    \item Scraper realiza a busca do dado;
    \item Scraper realiza a formatação do dado;
    \item O usuário abre o dado;
\end{enumerate}
\subsubsection{Fluxo Alternativo}
\begin{itemize}
    \item Dado não existe no 4chan: ocorre quando no passo 3 scraper não acha o dado indicado pelo usuário no 4chan.
    \begin{enumerate}
        \item Mostra mensagem de erro;
        \item Retorna ao passo 2.
    \end{enumerate}
    \item Scraper não reconhece o formato especificado: ocorre quando no passo 4 o scraper vai formatar o dado indicado pelo usuário e não consegue identificar o formato desejado.
    \begin{enumerate}
        \item Mostra mensagem de erro;
        \item Retorna ao passo 2.
    \end{enumerate}
\end{itemize}

\subsection{Baixar conteúdo de mídia de uma Thread do 4chan}
\subsubsection{Descrição}
Permite ao usuário baixar todo a mídia de uma thread do 4chan. Para isso o usuário deve passar como parâmetro para o \textit{Scraper} a id da thread que deseja baixar a mídia.
\subsubsection{Atores : Usuário}
\subsubsection{Pré-Condição}
O usuário deve informar o id das threads que deseja salvar.
\subsubsection{Pós-Condição}
O usuário terá o todo conteúdo de mídia(fotos,vídeos) da thread indicada.
\subsubsection{Fluxo Principal}
\begin{enumerate}
    \item Abre o Scraper pela linha de comando;
    \item Passa como parâmetro o id do dado que deseja e o formato dos dados;
    \item Scraper realiza a busca do dado;
    \item Scraper realiza a formatação do dado;
    \item O usuário abre o dado;
\end{enumerate}
\subsubsection{Fluxo Alternativo}
\begin{itemize}
    \item Dado não existe no 4chan: ocorre quando no passo 3 scraper não acha o dado indicado pelo usuário no 4chan.
    \begin{enumerate}
        \item Mostra mensagem de erro;
        \item Retorna ao passo 2.
    \end{enumerate}
    \item Scraper não reconhece o formato especificado: ocorre quando no passo 4 o scraper vai formatar o dado indicado pelo usuário e não consegue identificar o formato desejado.
    \begin{enumerate}
        \item Mostra mensagem de erro;
        \item Retorna ao passo 2.
    \end{enumerate}
\end{itemize}

\subsection{Baixar todo o conteúdo de uma Thread do 4chan}
\subsubsection{Descrição}
Permite ao usuário baixar todo a conteúdo de uma thread do 4chan (mídia e texto). Para isso o usuário deve passar como parâmetro para o \textit{Scraper} a id da thread que deseja baixar todo o conteúdo.
\subsubsection{Atores : Usuário}
\subsubsection{Pré-Condição}
O usuário deve informar o id das threads que deseja salvar.
\subsubsection{Pós-Condição}
O usuário terá o todo conteúdo da thread indicada.
\subsubsection{Fluxo Principal}
\begin{enumerate}
    \item Abre o Scraper pela linha de comando;
    \item Passa como parâmetro o id do dado que deseja e o formato dos dados;
    \item Scraper realiza a busca do dado;
    \item Scraper realiza a formatação do dado;
    \item O usuário abre o dado;
\end{enumerate}
\subsubsection{Fluxo Alternativo}
\begin{itemize}
    \item Dado não existe no 4chan: ocorre quando no passo 3 scraper não acha o dado indicado pelo usuário no 4chan.
    \begin{enumerate}
        \item Mostra mensagem de erro;
        \item Retorna ao passo 2.
    \end{enumerate}
    \item Scraper não reconhece o formato especificado: ocorre quando no passo 4 o scraper vai formatar o dado indicado pelo usuário e não consegue identificar o formato desejado.
    \begin{enumerate}
        \item Mostra mensagem de erro;
        \item Retorna ao passo 2.
    \end{enumerate}
\end{itemize}


\subsection{Especificar Threads para monitoramento}
\subsubsection{Descrição}
Permite ao usuário informar uma thread para monitoramento no 4chan, assim o scraper fará consultas constantemente, buscando e baixando alterações na thread. Para isso o usuário deve passar como parâmetro para o \textit{Scraper} a id da thread que deseja monitorar. 
\subsubsection{Atores : Usuário}
\subsubsection{Pré-Condição}
O usuário deve informar o id das threads que deseja monitorar.
\subsubsection{Pós-Condição}
O usuário terá o conteúdo da thread constantemente atualizado.
\subsubsection{Fluxo Principal}
\begin{enumerate}
    \item Abre o Scraper pela linha de comando;
    \item Passa como parâmetro o id da thread que deseja monitorar;
    \item Scraper realiza a busca da thread;
    \item Scraper começa o monitoramento da thread;
\end{enumerate}
\subsubsection{Fluxo Alternativo}
\begin{itemize}
    \item Thread não existe no 4chan: ocorre quando no passo 3 scraper não acha a thread indicado pelo usuário no 4chan.
    \begin{enumerate}
        \item Mostra mensagem de erro;
        \item Retorna ao passo 2.
    \end{enumerate}
\end{itemize}


\subsection{Especificar Boards para monitoramento}
\subsubsection{Descrição}
Permite ao usuário informar um board para monitoramento no 4chan, assim o scraper fará consultas constantemente, buscando e baixando alterações no board. Para isso o usuário deve passar como parâmetro para o \textit{Scraper} a id do board que deseja monitorar.  
\subsubsection{Atores : Usuário}
\subsubsection{Pré-Condição}
O usuário deve informar o id das boards que deseja monitorar.
\subsubsection{Pós-Condição}
O usuário terá o conteúdo do board constantemente atualizado.
\subsubsection{Fluxo Principal}
\begin{enumerate}
    \item Abre o Scraper pela linha de comando;
    \item Passa como parâmetro o id do board que deseja monitorar;
    \item Scraper realiza a busca do board;
    \item Scraper começa o monitoramento do board;
\end{enumerate}
\subsubsection{Fluxo Alternativo}
\begin{itemize}
    \item Board não existe no 4chan: ocorre quando no passo 3 scraper não acha o board indicado pelo usuário no 4chan.
    \begin{enumerate}
        \item Mostra mensagem de erro;
        \item Retorna ao passo 2.
    \end{enumerate}
\end{itemize}


\subsection{Interromper o monitoramento de uma Thread que estiver fechada}
\subsubsection{Descrição}
Permite ao usuário configurar o \textit{Scraper} para parar de monitorar threads que estiverem fechadas.
\subsubsection{Atores : Usuário}
\subsubsection{Pré-Condição}
Nenhuma pré-condição é necessária.
\subsubsection{Pós-Condição}
O scraper deixará de buscar atualizações na thread.
\subsubsection{Fluxo Principal}
\begin{enumerate}
    \item Abre o Scraper pela linha de comando;
    \item Passa como parâmetro o id da thread;
    \item Scraper adiciona ao monitoramento da thread a flag de fechada ou aberto;
\end{enumerate}
\subsubsection{Fluxo Alternativo}
\begin{itemize}
    \item Thread não existe no 4chan: ocorre quando no passo 3 scraper não acha a Thread indicado pelo usuário no 4chan.
    \begin{enumerate}
        \item Mostra mensagem de erro;
        \item Retorna ao passo 2.
    \end{enumerate}
    \item Scraper não monitora essa thread: ocorre quando no passo 3 scraper não tem essa Thread na sua lista de monitoração.
    \begin{enumerate}
        \item Mostra mensagem de erro;
        \item Retorna ao passo 2.
    \end{enumerate}
\end{itemize}


\subsection{Procurar Conteúdo pela Rede}
\subsubsection{Descrição}
Permite ao usuário procurar conteúdo pela rede peer-to-peer, para isso ele deve passar o nome ou hash do arquivo que deseja.
\subsubsection{Atores : Usuário}
\subsubsection{Pré-Condição}
O usuário deve informar o nome ou a hash do arquivo que deseja buscar.
\subsubsection{Pós-Condição}
O usuário saberá se um dado existe na rede, e os nodos que contem aquele dado.
\subsubsection{Fluxo Principal}
\begin{enumerate}
    \item Se conecta a rede pela linha de comando;
    \item Passa como parâmetro o nome ou hash do arquivo desejado;
    \item A rede pesquisa o arquivo pela rede;
    \item A rede informa uma lista de usuário que contém o arquivo;
\end{enumerate}
\subsubsection{Fluxo Alternativo}
\begin{itemize}
    \item O arquivo não existe na rede: ocorre quando no passo 3 a rede não acha o arquivo indicado pelo usuário.
    \begin{enumerate}
        \item Mostra mensagem de erro;
        \item Retorna ao passo 2.
    \end{enumerate}
\end{itemize}


\subsection{Requisitar Conteúdo pela Rede}
\subsubsection{Descrição}
Permite ao usuário requisitar conteúdo pela rede peer-to-peer, para isso ele deve passar o nome ou hash do arquivo que deseja.
\subsubsection{Atores : Usuário}
\subsubsection{Pré-Condição}
O usuário deve informar o nome ou a hash do arquivo que deseja requisitar.
\subsubsection{Pós-Condição}
O usuário terá o arquivo requisitado.
\subsubsection{Fluxo Principal}
\begin{enumerate}
    \item Se conecta a rede pela linha de comando;
    \item Passa como parâmetro o nome ou hash do arquivo desejado;
    \item A rede pesquisa o arquivo pela rede;
    \item A rede informa uma lista de usuário que contém o arquivo;
    \item O usuário informa se conecta a um dos usuários da lista;
    \item O usuário baixa o arquivo desejado;
\end{enumerate}
\subsubsection{Fluxo Alternativo}
\begin{itemize}
    \item O arquivo não existe na rede: ocorre quando no passo 3 a rede não acha o arquivo indicado pelo usuário.
    \begin{enumerate}
        \item Mostra mensagem de erro;
        \item Retorna ao passo 2.
    \end{enumerate}
    \item Não existe peers Confiáveis: ocorre quando no passo 4 nenhum dos peers está na lista de confiáveis do usuário.
    \begin{enumerate}
        \item Mostra mensagem se o usuário deseja prosseguir com um peer não confiável;
        \item Segue ao passo 5 ou encerra a execução.
    \end{enumerate}
\end{itemize}


\subsection{Fornecer Conteúdo pela Rede}
\subsubsection{Descrição}
Permite ao usuário fornecer conteúdo pela rede peer-to-peer, para isso ele deve ter o arquivo e configurar a rede para fornecer o conteúdo.
\subsubsection{Atores : Usuário}
\subsubsection{Pré-Condição}
O usuário deve ter armazenado os arquivos que deseja fornecer.
\subsubsection{Pós-Condição}
O usuário estará disponível na para outros usuários quando pesquisarem pelo arquivo servido. 
\subsubsection{Fluxo Principal}
\begin{enumerate}
    \item Se conecta a rede pela linha de comando;
    \item Passa como parâmetro o local do arquivo em sua maquina;
    \item A rede lê o arquivo;
    \item A rede gera uma hash e espoem o arquivo para a rede;
\end{enumerate}
\subsubsection{Fluxo Alternativo}
\begin{itemize}
    \item O arquivo não existe no caminho informado: ocorre quando no passo 3 a rede não acha o arquivo indicado pelo usuário.
    \begin{enumerate}
        \item Mostra mensagem de erro;
        \item Retorna ao passo 2.
    \end{enumerate}
\end{itemize}


\subsection{Criar um par de chaves criptográficas}
\subsubsection{Descrição}
Permite ao usuário criar um par de chaves criptográficas para garantir a autenticidade e integridade no compartilhamento de arquivos na rede.
\subsubsection{Atores : Usuário}
\subsubsection{Pré-Condição}
Nenhuma pré-condição é necessária.
\subsubsection{Pós-Condição}
O usuário terá um chave pública e uma chave criptográfica.
\subsubsection{Fluxo Principal}
\begin{enumerate}
    \item Se conecta ao modulo de criptografia pela linha de comando;
    \item Executa o método para criação de chaves criptográficas;
    \item O modulo de criptografia gera um par de chaves criptográficas para o usuário;
\end{enumerate}


\subsection{Criar lista de nodos confiáveis}
\subsubsection{Descrição}
Permite ao usuário criar uma lista de nodos confiáveis, para isso o usuário deve armazenar uma lista com a chave pública dos nodos que julgar confiáveis.
\subsubsection{Atores : Usuário}
\subsubsection{Pré-Condição}
O usuário deve ter no mínimo uma chave publica de um nodo para criar a lista.
\subsubsection{Pós-Condição}
O usuário terá uma lista de nodos confiáveis que poderá ser usado para verificar a identidade de um usuário e auxiliar no download de dados pela rede.
\subsubsection{Fluxo Principal}
\begin{enumerate}
    \item Se conecta a rede pela linha de comando;
    \item Executa o método para criação de lista de nodos confiáveis passando como parâmetro a chave publica de um nodo confiável;
    \item A rede gera a lista de nodos confiáveis;
\end{enumerate}
\subsubsection{Fluxo Alternativo}
\begin{itemize}
    \item A lista já foi criada: ocorre quando no item 3 já ter uma lista criada.
    \begin{enumerate}
        \item Mostra mensagem de erro, perguntado se deseja substituir a lista;
        \item Substitui a lista ou encerra o programa.
    \end{enumerate}
\end{itemize}


\subsection{Exportar minha chave pública para Compartilhamento}
\subsubsection{Descrição}
Permite ao usuário exportar a chave pública no formato cleartext para compartilhamento. Para isso o usuário necessita ter um par de chaves criptográficas. 
\subsubsection{Atores : Usuário}
\subsubsection{Pré-Condição}
O usuário precisa ter chaves criptográficas.
\subsubsection{Pós-Condição}
O usuário terá sua chave publica num formato de fácil compartilhamento.
\subsubsection{Fluxo Principal}
\begin{enumerate}
    \item Se conecta a rede pela linha de comando;
    \item Executa o método para exportar a chave publica no formato cleartext;
    \item A rede gera a converte a chave publica para cleartext e informa ao usuário;
\end{enumerate}
\subsubsection{Fluxo Alternativo}
\begin{itemize}
    \item O usuário não tem chave publica cadastrada: ocorre quando no item 3 o usuário não tem um par de chaves criptográficas criadas.
    \begin{enumerate}
        \item Mostra mensagem de erro;
        \item Encerra o programa.
    \end{enumerate}
\end{itemize}


\subsection{Assinar arquivos com a minha chave privada}
\subsubsection{Descrição}
Permite ao usuário assinar arquivos com sua chave privada para que os outros peers da rede que o mantém como nodo confiável e tem sua chave publica possam reconhecer a assinatura e identificar que ele é um peer confiável. Para isso o usuário deve ter um par de chaves criptográficas.
\subsubsection{Atores : Usuário}
\subsubsection{Pré-Condição}
O usuário precisa ter chaves criptográficas.
\subsubsection{Pós-Condição}
O arquivo poderá ser verificado por usuários que tenham a chave pública correspondente.
\subsubsection{Fluxo Principal}
\begin{enumerate}
    \item Se conecta a rede pela linha de comando;
    \item Executa o método para assinar um arquivo passando como parâmetro o caminho do arquivo na maquina do usuário;
    \item A rede gera uma assinatura no arquivo;
\end{enumerate}
\subsubsection{Fluxo Alternativo}
\begin{itemize}
    \item O usuário não tem chave publica cadastrada: ocorre quando no item 3 o usuário não tem um par de chaves criptográficas criadas.
    \begin{enumerate}
        \item Mostra mensagem de erro;
        \item Encerra o programa.
    \end{enumerate}
     \item O arquivo não existe: ocorre quando no item 3 o programa não acha o arquivo no caminho especificado pelo usuário.
    \begin{enumerate}
        \item Mostra mensagem de erro;
        \item Retorna ao passo 2.
    \end{enumerate}
\end{itemize}


\subsection{Verificar se a assinatura de um arquivo pertence a minha lista de nodos confiáveis}
\subsubsection{Descrição}
Permite ao usuário identificar a assinatura de um nodo confiável a partir da chave publica na lista de nodos confiáveis. Para isso o usuário necessita ter armazenado na sua lista de nodos confiáveis a chave publica do nodo.
\subsubsection{Atores : Usuário}
\subsubsection{Pré-Condição}
O usuário precisa ter uma lista de nodos confiáveis.
\subsubsection{Pós-Condição}
O usuário terá verificado a identidade de quem assinou aquele arquivo.
\subsubsection{Fluxo Principal}
\begin{enumerate}
    \item Se conecta a rede pela linha de comando;
    \item Executa o método para buscar um arquivo na passando como parâmetro o nome ou a hash do arquivo;
    \item A rede pesquisa o arquivo pela rede;
    \item A rede informa uma lista de usuário que contém o arquivo;
    \item O usuário seleciona o arquivo e usuário e pede para rede verificar a assinatura;
    \item A rede verifica a assinatura.
\end{enumerate}
\subsubsection{Fluxo Alternativo}
\begin{itemize}
    \item O arquivo não existe na rede: ocorre quando no passo 3 a rede não acha o arquivo indicado pelo usuário.
    \begin{enumerate}
        \item Mostra mensagem de erro;
        \item Retorna ao passo 2.
    \end{enumerate}
\end{itemize}


\chapter{Metodologia}

A metodologia empregada para o planejamento e desenvolvimento do sistema deste trabalho de conclusão é o tradicional modelo cascata.
Esse modelo funciona como um exemplo de um processo dirigido a planos, onde, em princípio deve-se planejar e programar todas as atividades do processo até a implementação e a manutenção do dele~\cite{SOMMERVILLE1}.

Consiste numa sequência de fases em que cada fase dá suporte para a próxima, assim o produto de cada fase torna-se a entrada para a seguinte~\cite{BALTZAN1}, a fase inicial é definição de requisitos que busca estabelecer os serviços, restrições e metas do sistema, normalmente por meio de consultas com o usuário~\cite{SOMMERVILLE1}; a próxima fase projeto de sistema e software é a fase em que a identificação e descrição das abstrações fundamentais do sistema é realizada, alocando os requisitos do sistema por meio de uma arquitetura geral do sistema~\cite{SOMMERVILLE1}; a terceira fase implementação e teste unitário é a fase de desenvolvimento do projeto como um conjunto de programas ou unidades de programa, utilizando o teste unitário para verificar de cada unidade, tendo certeza de que a mesma cumpra sua especificação~\cite{SOMMERVILLE1}; a quarta fase integração e teste de sistema é quando as unidades criadas na fase anterior de maneira independente são integradas e testadas como um sistema único e completo, verificando se todos os requisitos estão sendo atendidos~\cite{SOMMERVILLE1}.
Na fase final, Operação e manutenção, é o momento em que o sistema é de fato instalado e colocado em uso, é onde os eventuais erros que não apareceram anteriormente são resolvidos e, também, onde novos requisitos podem ser descobertos~\cite{SOMMERVILLE1}. 
\begin{figure}[H]
    \centering
    \includesvg[width=\textwidth]{fig/Modelo_Cascata.svg}
    \caption[Diagrama do Modelo Cascata]{\label{fig:Modelo_Cascata}
        Diagrama do Modelo Cascata\\
        Fonte: os autores.
    }
\end{figure}
Por mais que essa seja uma metodologia não mais tão utilizada pelas empresas devido à sua característica de precisão extrema que não leva em consideração mudanças durante o desenvolvimento do projeto~\cite{BALTZAN1}, para este trabalho ela se torna uma boa escolha, pois se trata de um Trabalho de Conclusão de Curso, com um único desenvolvedor e prazos muito bem definidos.
Devido à inexistência de um cliente e visando um público alvo, reuniões com o cliente são impossíveis e testes durante o desenvolvimento do trabalho com alguém que faça parte do público alvo não acontecerão facilmente, o modelo cascata se torna a melhor opção como metodologia de desenvolvimento do projeto. A partir de agora as fases alvo são: a codificação, os testes e a implementação/manutenção do sistema, concluindo assim o ciclo completo do modelo tradicional em cascata.
\chapter{Cronograma}

Apresentamos a seguir um cronograma relatando nossas atividades desenvolvidas e nosso planejamento, relatando as atividades que já foram e que serão desenvolvidas.

\begin{enumerate}
    \item \label{cron:search} Seleção e análise das redes sociais atuais.
    \item \label{cron:background} Consolidação da fundamentação teórica.
    \item \label{cron:architecture} Elaboração da proposta de Arquitetura.
    \item \label{cron:proposal} Entrega da proposta de TC.
    \item \label{cron:requirements} Identificação dos requisitos do sistema.
    \item \label{cron:diagrams} Elaboração da modelagem do sistema.
    \item \label{cron:methodology} Definição da metodologia de desenvolvimento.
    \item \label{cron:esc-tcI}  Entrega do volume final de TC1.
    \item \label{cron:scraper} Desenvolvimento do \textit{Scraper}.
    \item \label{cron:val1}  Teste e validação do \textit{Scraper} e estrutura de dados.
    \item \label{cron:network} Desenvolvimento da rede.
    \item \label{cron:val2} Teste e validação da rede e do protocolo da rede.
    \item \label{cron:integration} Integração entre \textit{Scraper} e rede de dados.
    \item \label{cron:val3} Testes da Integração.
    \item \label{cron:gui} Desenvolvimento da interface gráfica.
    \item \label{cron:val4} Testes e validações finais do sistema.
    \item \label{cron:poster} Entrega dos Cartazes.
    \item \label{cron:esc-tcII} Entrega da volume final de TC2.
\end{enumerate}

\definecolor{midgray}{gray}{.5}
\begin{table}[!htbp]
    \centering
    \begin{tabular}{|c|c|c|c|c|c|}
        \hline
                                & \multicolumn{5}{c|}{2021}                                                                                         \\
        \hline
                                & MAR                       & ABR                 & MAI                 & JUN                 & JUL                 \\
        \hline
        \ref{cron:search}       & \cellcolor{midgray}       &                     &                     &                     &                     \\
        \hline
        \ref{cron:background}   & \cellcolor{midgray}       &                     &                     &                     &                     \\
        \hline
        \ref{cron:architecture} &                           & \cellcolor{midgray} &                     &                     &                     \\
        \hline
        \ref{cron:proposal}     &                           & \cellcolor{midgray} &                     &                     &                     \\
        \hline
        \ref{cron:requirements} &                           & \cellcolor{midgray} &                     &                     &                     \\
        \hline
        \ref{cron:diagrams}     &                           & \cellcolor{midgray} &                     &                     &                     \\
        \hline
        \ref{cron:methodology}  &                           & \cellcolor{midgray} &                     &                     &                     \\
        \hline
        \ref{cron:esc-tcI}      &                           & \cellcolor{midgray} &                     &                     &                     \\
        \hline
        \ref{cron:scraper}      &                           &                     & \cellcolor{midgray} &                     &                     \\
        \hline
        \ref{cron:val1}         &                           &                     & \cellcolor{midgray} &                     &                     \\
        \hline
        \ref{cron:network}      &                           &                     & \cellcolor{midgray} &                     &                     \\
        \hline
        \ref{cron:val2}         &                           &                     & \cellcolor{midgray} &                     &                     \\
        \hline
        \ref{cron:integration}  &                           &                     &                     & \cellcolor{midgray} &                     \\
        \hline
        \ref{cron:val3}         &                           &                     &                     & \cellcolor{midgray} &                     \\
        \hline
        \ref{cron:gui}          &                           &                     &                     & \cellcolor{midgray} &                     \\
        \hline
        \ref{cron:val4}         &                           &                     &                     & \cellcolor{midgray} &                     \\
        \hline
        \ref{cron:poster}       &                           &                     &                     & \cellcolor{midgray} &                     \\
        \hline
        \ref{cron:esc-tcII}     &                           &                     &                     &                     & \cellcolor{midgray} \\
        \hline
    \end{tabular}
    \caption{\label{tab:schedule}
        Cronograma de atividades.\\
        Fonte: os autores.
    }
\end{table}



\chapter{Conclusão}

\section{Limitações}

Usuário 

\section{Trabalhos futuros}

\subsection{Sistema de governança}

A definição dos formatos e estruturas de dados precisa ser gerida de forma aberta, transparente, e não-monopolística.

Extensões do scraper (ou outros scrapers) para baixar conteúdos encontrados em links.
