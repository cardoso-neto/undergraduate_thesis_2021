
\chapter{Referencial teórico}

\section{Mídia Alternativa}
Esta seção cobre trabalhos referentes à mídia alternativa de tal forma a construir histórico e  contexto até o início das redes sociais, definindo o que é mídia alternativa  e os impactos que as atuais redes sociais causam e causaram nas mídias alternativas.
\subsection{Benjamin, W. (1970). The author as producer.}
Benjamin Walter em seu texto “o autor como produtor”, constrói alguns conceitos importantes para se entender a mídia alternativa. Como sugerido pelo título, o autor explica a importância de o autor participar da produção da mídia, para isso usa como exemplo o filósofo Platão e seu livro ‘A República’. Platão em seu projeto de Estado não permite que os poetas ali vivam, mesmo considerando o grande poder da poesia, nessa comunidade perfeita este poder seria destrutivo e supérfluo. Desde Platão o direito de existir do poeta não foi constantemente debatido, até o século 20. Benjamim insinua que da mesma forma que Platão queria limitar a liberdade do poeta, hoje ela está sendo limitada, não no modo radical de Platão de debater a existência do poeta, mas na limitação da autonomia do poeta: sua liberdade de escrever o que quiser.

A Situação social atual obriga o poeta a escolher a quem sua atividade servirá, ele trabalha no interesse de uma determinada classe. Mas então sua autonomia está acabada. Dizemos que ele defende uma tendência. Então o autor toma uma posição em relação a mídia e propaganda. Em vez de apenas reproduzir um argumento como conteúdo em uma publicação, seguindo uma tendência, Benjamin sustentou que, para a propaganda ser eficaz, o próprio meio exigia uma transformação: oposição do trabalho em relação aos meios de produção tinha que ser criticamente realinhado, não apenas o argumento na página. Isto exige não apenas a radicalização dos métodos de produção, mas um repensar dos o que significa ser um produtor de mídia.

\subsection{Atton, C. (2002). Alternative media.}
Atton em seu livro Alternative media, defende a democratização da comunicação definindo o principal dever da mídia alternativa, "A mídia alternativa tem que fornecer os meios para uma comunicação democrática para as pessoas que são normalmente excluídas da produção de mídia", estabelecendo assim que a mídia alternativa é definida não só pelo seu conteúdo, mas também pelas práticas adotadas na sua produção, que deve permitir a participação livre e igualitária na produção da mídia. A teoria da mídia alternativa é assim, intimamente ligada às teorias da democracia radical: a equalização das hierarquias de poder. Assim como Benjamin e utilização de premissas semelhantes, Atton defende a participação direta do intermediador no meio ao qual ele está debatendo, não sendo apenas um observador terceiro. 

\subsection{Nick Couldry}
O Ator reflete acerca do poder da mídia e o problema que a concentração desse poder. Segundo Couldry é impossível isolar o ‘poder midiático' de conflitos sociais, isso vai contra a maioria das teorias sociais e midiáticas, porém é importante para uma análise comparativa para a mídia alternativa e seu potencial de crescimento.
Mídia constantemente registra a influência de várias forças externas (estado e influências corporativas por exemplo), mesmo assim ela representa uma força paralela a essas, com o poder e dever de contestação, representando o poder simbólico. Couldry põem esse poder simbólico como um dos poderes fundamentais (econômico, político, militar e simbólico). Para explicar o poder simbólico é citando Thompson 1995:17.
\begin{directcite}
Symbolic power is the capacity to intervene in the course of events, to influence the 
Actions of others and indeed create events, by means of the production and transmission of symbolic form.
\end{directcite}

A concentração esmagadora de poder simbólico, com foco na mídia pode ser definida como o poder de ‘construir realidades’, isso é a realidade da sociedade. Portanto, contestar a mídia nessa situação seria contestar o jeito que a realidade social é definida.
Por esse motivo a mídia só pode ser contestada por alternativas periféricas a essa realidade social sendo isso uma característica essencial  para garantir a sustentabilidade  do poder fundamental simbólico, e  a principal forma de mudança social feita pela mídia.
O autor alega que para contestar a mídia deve-se tomar iniciativas fora da infraestrutura tradicional de mídia. Para  isso ele  cita o ativista de software Matthew Arnison:

\begin{directcite}
“A tecnologia da velha mídia cria uma hierarquia natural entre os contadores de histórias e o público. O contador de histórias tem acesso a alguma tecnologia, como um transmissor de TV ou impressora. O público não. . .Em algum lugar ao longo do caminho, isso foi justificado assumindo que a maioria das pessoas não são tão criativos, ter apenas um punhado de pessoas para contar histórias em uma cidade de milhões é uma maneira natural de fazer as coisas. Mas é isso? . . .” (Arnison, 2002a: 1)
\end{directcite}

O Autor ataca e assume como problema o fato de poucas pessoas produzirem mídia e muitas pessoas consumirem, resultando numa concentração de poder simbólico nociva. Para explicar o por que é natural essa concentração, mesmo nociva, de poder simbólico, é usado o fator que é necessário muito poder econômico para montar e manter uma infraestrutura e conseguir escalar.

Assim, o autor fornece quatro tradicionais exemplos que podem reunir  poder econômico necessário para contestar a mídia tradicional: Corporações, estado, exército e religião.
\begin{itemize}
    \item As corporações não são uma fonte promissora de alternativas ao poder da mídia, justamente porque o negócio de venda está intimamente ligado à manutenção do acesso ao mercado, que por sua vez, em todas as sociedades contemporâneas, depende do alcance das instituições de mídia;
    \item Os estados são, a primeira vista, mais promissores, pelo menos como uma fonte de subsídio para aqueles que querem construir alternativas ao poder da mídia: primeiro, porque os estados têm seus próprios recursos simbólicos bem estabelecidos (para regular as fronteiras do estado e as definições de cidadania, para controlar o termos nos quais as empresas podem ou não operar, mercadorias e imagens podem circular). Em segundo lugar, estados específicos podem às vezes sentir que seus interesses estão em desacordo com as agendas das instituições de mídia. As ocasiões em que (com ou sem os militares, o que podemos evitar considerar separadamente, visto que geralmente carecem de recursos simbólicos próprios) os estados contestaram o poder da mídia dificilmente serão comemoradas. Por outro lado, a história da mídia do século 20 oferece outros exemplos de como o interesse do estado moderno no alcance retórico da mídia emergente (como a BBC na Grã-Bretanha) subsidiou alternativas institucionais para completar a centralização da mídia (por exemplo, a programação regional da BBC, o Movimento de 'acesso' à televisão dos anos 1960/70). Sem esses subsídios estatais, não poderia ter surgido a relativamente equilibrada ‘ecologia da mídia’ em países como a Grã-Bretanha, com sua tradição de serviço público. No entanto, à medida que diminui a capacidade do estado de influenciar as estruturas do mercado global, ele se torna um rival cada vez mais precário para o poder da mídia, especialmente quando o poder do estado, como o poder corporativo, depende cada vez mais do acesso da mídia aos mercados (geralmente chamados de "eleitorados");
    \item Na maioria das sociedades, as instituições religiosas promulgam suas próprias narrativas de enquadramento do mundo social e, na verdade, do cosmos, que não dependem diretamente dos pontos de referência da mídia. O papel da crença religiosa, como um local de desafios para o 'enquadramento' da mídia, é um dos tópicos mais negligenciados nos estudos de mídia, embora após 11 de setembro de 2001 sua negligência seja difícil de defender.O autor ainda cita a importância da igreja católica para a mídia alternativa em comunidades na América Latina;
\end{itemize}

O Autor então propõe uma cisão nessa análise dos meios tradicionais de mídia alternativa e para dar um enfoque a novas questões que devem ser analisadas visando o futuro da mídia alternativa. Assim ele cita três dimensões, as novas formas de consumo da mídia, as novas infraestruturas de produção e as novas infraestruturas de distribuição.  

Na primeira dimensão, consumo, Couldry indica que a forma de mensurar audiência é um tanto obscura, citando que uma boa parte do acesso a mídia pode ser caracterizada como acesso privado, citando o hábito de algumas pessoas de receber notícias de uma variedade mutável de fontes localizadas da web (incluindo redes semi públicas e testemunhos privados), e de quanta credibilidade essas fontes terão em comparação a mídia tradicional.

Isso leva à segunda e terceira dimensões, dimensões que são menos familiares, uma vez que as mudanças infraestruturais são sempre difíceis de isolar. A Internet aumentou drasticamente nosso interesse pela infraestrutura de mídia: primeiro, porque a Internet aumentou a facilidade com que qualquer material digitalizado pode ser distribuído além das fronteiras nacionais, organizacionais e sociais; segundo, porque novas formas de software de código aberto estão aumentando a velocidade com que as inovações na produção digital podem se espalhar, pelo menos entre aqueles com alto conhecimento de informática. Os impactos de longo prazo dessas mudanças no consumo da mídia e as crenças das pessoas sobre o status social da mídia permanecem incertos, mas é, eu sugiro, para novas formas híbridas de produtor-consumidor de mídia que devemos buscar mudanças, uma vez que elas desafiam precisamente os arraigados divisão de trabalho (produtor de histórias versus consumidor de histórias) que é a essência do poder da mídia.

Couldry acredita que para provocar uma real mudança no poder midiático deve envolver todas as dimensões do processo de criação de mídia (produção, distribuição, consumo e infraestrutura).

Por fim o autor conclui propondo que o crescente uso da internet como plataforma de mídia pode ser e é o futuro da mídia alternativa.
\subsection{Clemência Rodrigues}
Ao debater o futuro da pesquisa da mídia, analisa o impacto das redes sociais e plataformas da web 2.0 na mídia e em movimentos sociais.

Clemência faz um levantamento histórico da relação mídia corporativa e movimentos sociais, mostrando uma relação conturbada entre as duas partes, através de tentativas por parte de movimentos sociais de se apropriar de tecnologia da mídia corporativa. As redes sociais da web 2.0 mudam esse paradigma de confronto, assim movimentos sociais passaram a usar plataformas corporativas como facebook, twitter, youtube etc…

Segundo a autora as redes sociais tornaram um gigantesco espaço na pesquisa midiática, havendo assim movimentos sociais sem nomeados como “revoluções sem líderes”, ou até mesmo citando Mark Zuckerberg como o líder da revolução (citação), depois de 2009 uma manifestação de estudantes, ficou conhecida como “the twitter revolution”. Toda essa importância dada às mídias sociais negligenciou os movimentos sociais, fazendo com que movimentos que cresceram ao longo dos anos 2000 no oriente médio e se concretizaram na primavera árabe fossem esquecidos e os créditos da primavera árabe fossem dados ao twitter. 

O grande crédito e influência dados as plataformas sociais corporativas encobriram  problemas como a alta influência que as redes sociais sofrem dos poderes políticos e econômicos, já que a infraestrutura de comunicação e tecnologia é moldada por regimes regulatórios, acordos internacionais, ganancias corporativas e práticas intrusivas de vigilância. Assim segundo Sasha Costanza-Chock:

\begin{directcite}
‘in business-speak, “User Generated Content” means free cultural product for monetization and cross-licensing, “participation” means free user data to mine and sell to advertisers, and all user activity is subject to surveillance and censorship’ 
\end{directcite}

\section{PoVs}
% pontos de vista sobre o problema

\section{Related Work}
% alternativas ao que proporemos

\subsection{Non-indexed Scraping-based Archives}
% Background teórico
4plebs[citar] is the most famous 4chan scraper and makes its data available[citar] through tar.gz archives hosted on archive.org[citar], as well as a RESTful API[citar] and a 4chan-like front-end[citar] hosted at the owner's cost.

\subsection{Pay-to-use}

Tim Berners-Lee's Solid
zeronet
Scuttlebutt


\section{The Architecture of the Web}

\subsection{Web 1.0}
Static web-pages.
Easily cloneable with wget.

\subsection{Web 2.0}
Many users interacting with few corporate-owned centralized servers.

\subsection{Web 3.0}
Distributed, decentralized, and permanent.

\section{Great Tech We Will Build Upon}
% tech that makes what we'll propose possible
\subsection{Web-scraping}

\subsection{Digital Signatures}

\subsection{Cryptographic Hash Functions}

\subsection{IPFS}

\chapter{Solutions}

\section{Partial Solutions}

\subsection{Facebook migrating to a more web 3.0 architecture}

\section{Our Proposal}

\chapter{Proof of Concept}

\section{Making 4chan Great Again}

\section{Implementation details}

\chapter{Final Thoughts}
